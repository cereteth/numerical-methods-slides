% Copyright Luke Olson 2009--2014
% This work is licensed under the Creative Commons
% Attribution-NonCommercial-NoDerivatives 4.0 International License. To view a
% copy of this license, visit http://creativecommons.org/licenses/by-nc-nd/4.0/.
%
\documentclass[10pt]{beamer}
%\documentclass[handout,10pt]{beamer}
%
\mode<presentation>
{
  \usetheme[secheader]{Boadilla}
  \usecolortheme{luke}
  \usefonttheme[onlymath]{serif}
  \setbeamercovered{invisible}
  %\setbeamercovered{transparent}
  %
}
\mode<handout>
{
  \usetheme[secheader]{Boadilla}
  \usefonttheme[onlymath]{serif}
  \setbeamercovered{invisible}
  \usecolortheme{luke2}
  %\setbeamercovered{transparent}
}
\usepackage{pgf,pgfarrows,pgfnodes,pgfautomata,pgfheaps,pgfshade}
\usepackage{pxfonts}
\usepackage{pifont}
\usepackage{eulervm}
\usepackage{listings}
%\usepackage{pgfpages}
%\pgfpagesuselayout{2 on 1}[letterpaper]
%
%
%%%%%%%%%%%%%%%%%%%%%%%%%%%%%%%%%%%%%%%%%%%%%%%%%%%%%%%%%%%%%%%%%%%%%%%%


%
%
%
\newcommand{\vb}{{\bf{b}}}
\newcommand{\ve}{{\bf{e}}}
\newcommand{\vg}{{\bf{g}}}
\newcommand{\vp}{{\bf{p}}}
\newcommand{\vr}{{\bf{r}}}
\newcommand{\vu}{{\bf{u}}}
\newcommand{\vx}{{\bf{x}}}
\newcommand{\vz}{{\bf{z}}}
\newcommand{\vA}{{\bf{A}}}
\newcommand{\vU}{{\bf{U}}}
\newcommand{\mO}{{\mathcal{O}}}
\newcommand{\mF}{{\mathcal{F}}}
\definecolor{mygray}{rgb}{0.95,0.95,0.95}
\lstset{
        language=matlab,
        numbers=left, numberstyle=\tiny, stepnumber=1, numbersep=5pt,
        basicstyle=\color{black}\ttfamily\small,
        commentstyle=\color{green}\ttfamily,
        keywordstyle=\color{blue}\ttfamily,
        stringstyle=\color{red}\ttfamily,
        showstringspaces=false,
        backgroundcolor=\color{mygray},
        breaklines,
}
\newcommand{\norm}[1]{{\ensuremath{{\|#1\|}}}}
\newcommand{\matdim}[2]{\ensuremath{#1\times#2}}
\newcommand{\rank}[1]{\ensuremath{\mathrm{rank}(#1)}}
\newcommand{\epsm}{\ensuremath{\varepsilon_m}}
\newcommand{\cmd}[1]{{\normalfont\ttfamily\bfseries#1}}

\author{L. Olson}
\institute[UIUC]
{Department of Computer Science\\
University of Illinois at Urbana-Champaign\\
\vspace{0.5cm}
}
%%%%%%%%%%%%%%%%%%%%%%%%%%%%%%%%%%%%%%%%%%%%%%%%%%%%%%%%%%%%%%%%%%%%%%%%
\pgfdeclareimage[height=0.5cm]{university-logo}{./figs/uiuclogo}
\logo{\pgfuseimage{university-logo}}
%%%%%%%%%%%%%%%%%%%%%%%%%%%%%%%%%%%%%%%%%%%%%%%%%%%%%%%%%%%%%%%%%%%%%%%%
\title[CS 357]{Lecture 5a}
\subtitle{Linear Algebra Recall}
\date{September 8, 2009}

\begin{document}
% -------------------------------------------------
\begin{frame}
  \titlepage
\end{frame}
% -------------------------------------------------
%%%%%%%%%%%%%%%%%%%%%%%%%%%%%%%%%%%%%%%%%%%%%%%%%%%%%%%%%%%%%%%%%%%%%%%%
\begin{frame}
\frametitle{Vector Operations}

\begin{itemize}
    \item   Addition and Subtraction
    \item   Multiplication by a scalar
    \item   Transpose
    \item   Linear Combinations of Vectors
    \item   Inner Product
    \item   Outer Product
    \item   Vector Norms
\end{itemize}

% ----------------------
\end{frame}
%%%%%%%%%%%%%%%%%%%%%%%%%%%%%%%%%%%%%%%%%%%%%%%%%%%%%%%%%%%%%%%%%%%%%%%%
%%%%%%%%%%%%%%%%%%%%%%%%%%%%%%%%%%%%%%%%%%%%%%%%%%%%%%%%%%%%%%%%%%%%%%%%
\begin{frame}
\frametitle{Vector Addition and Subtraction}

Addition and subtraction are element-by-element operations
\begin{align*}
    c = a + b\ \ \ &\Longleftrightarrow\ \ \ c_i = a_i + b_i\ \ \ i = 1,\ldots,n  \\
    d = a - b\ \ \ &\Longleftrightarrow\ \ \ d_i = a_i - b_i\ \ \ i = 1,\ldots,n
\end{align*}

%\vspace{3ex}
\vspace{0.0cm}
\begin{equation*}
    a = \begin{bmatrix} 1 \\ 2 \\ 3\end{bmatrix}
    \qquad
    b = \begin{bmatrix} 3 \\ 2 \\ 1\end{bmatrix}
\end{equation*}

\begin{equation*}
    a+b = \begin{bmatrix} 4 \\ 4 \\ 4 \end{bmatrix}
    \qquad
    a-b = \left[\negthickspace\begin{array}{r} -2 \\ 0 \\ 2 \end{array}\right]
\end{equation*}

% ----------------------
\end{frame}
%%%%%%%%%%%%%%%%%%%%%%%%%%%%%%%%%%%%%%%%%%%%%%%%%%%%%%%%%%%%%%%%%%%%%%%%
%%%%%%%%%%%%%%%%%%%%%%%%%%%%%%%%%%%%%%%%%%%%%%%%%%%%%%%%%%%%%%%%%%%%%%%%
\begin{frame}
\frametitle{Multiplication by a Scalar}

Multiplication by a scalar involves multiplying each element
in the vector by the scalar:
\begin{equation*}
    b = \sigma a\ \ \ \Longleftrightarrow\ \ \  b_i = \sigma a_i\ \ \ i = 1,\ldots,n
\end{equation*}

\begin{equation*}
    a = \begin{bmatrix} 4 \\ 6 \\ 8 \end{bmatrix}
    \qquad
    b = \frac{a}{2} = \begin{bmatrix} 2 \\ 3  \\ 4 \end{bmatrix}
\end{equation*}


% ----------------------
\end{frame}
%%%%%%%%%%%%%%%%%%%%%%%%%%%%%%%%%%%%%%%%%%%%%%%%%%%%%%%%%%%%%%%%%%%%%%%%
%%%%%%%%%%%%%%%%%%%%%%%%%%%%%%%%%%%%%%%%%%%%%%%%%%%%%%%%%%%%%%%%%%%%%%%%
\begin{frame}
\frametitle{Vector Transpose}

The \emph{transpose} of a row vector is a column vector:
\begin{equation*}
    u = \bigl[1, 2, 3\bigr]
    \qquad
    \text{then}
    \qquad
    u^T = \begin{bmatrix}1 \\ 2 \\ 3\end{bmatrix}
\end{equation*}
Likewise if $v$ is the column vector
\begin{equation*}
    v = \begin{bmatrix}4 \\ 5 \\ 6\end{bmatrix}
    \qquad
    \text{then}
    \qquad
    v^T = \bigl[4, 5, 6\bigr]
\end{equation*}


% ----------------------
\end{frame}
%%%%%%%%%%%%%%%%%%%%%%%%%%%%%%%%%%%%%%%%%%%%%%%%%%%%%%%%%%%%%%%%%%%%%%%%
%%%%%%%%%%%%%%%%%%%%%%%%%%%%%%%%%%%%%%%%%%%%%%%%%%%%%%%%%%%%%%%%%%%%%%%%
\begin{frame}
\frametitle{Linear Combinations}

Combine scalar multiplication with addition
\begin{equation*}
      \alpha \begin{bmatrix} u_1 \\ u_2 \\ \vdots \\ u_m\end{bmatrix}
    + \beta  \begin{bmatrix} v_1 \\ v_2 \\ \vdots \\ v_m\end{bmatrix}
    = \begin{bmatrix}\alpha u_1 + \beta v_1\\
                     \alpha u_2 + \beta v_2\\
                     \vdots \\
                     \alpha u_m + \beta v_m  \end{bmatrix}
    = \begin{bmatrix} w_1 \\ w_2 \\ \vdots \\ w_m \end{bmatrix}
\end{equation*}

\vspace{0.0cm}
\begin{equation*}
    r = \left[\negthickspace\begin{array}{r} -2 \\ 1 \\ 3 \end{array}\right]
    \qquad
    s = \begin{bmatrix} 1 \\ 0  \\ 3 \end{bmatrix}
\end{equation*}

\begin{equation*}
    t = 2r + 3s = \left[\negthickspace\begin{array}{r} -4 \\ 2 \\ 6 \end{array}\right]
                  + \begin{bmatrix} 3 \\ 0  \\ 9 \end{bmatrix}
                = \left[\negthickspace\begin{array}{r} -1 \\ 2 \\ 15 \end{array}\right]
\end{equation*}

% ----------------------
\end{frame}
%%%%%%%%%%%%%%%%%%%%%%%%%%%%%%%%%%%%%%%%%%%%%%%%%%%%%%%%%%%%%%%%%%%%%%%%
%%%%%%%%%%%%%%%%%%%%%%%%%%%%%%%%%%%%%%%%%%%%%%%%%%%%%%%%%%%%%%%%%%%%%%%%
\begin{frame}[shrink]
\frametitle{Linear Combinations}

Any one vector can be created from an infinite combination
of other ``suitable'' vectors.

\begin{center}
\begin{tabular}{cl}
    $w=$ &  $\displaystyle\begin{bmatrix} 4 \\ 2 \end{bmatrix}
           = 4\begin{bmatrix} 1 \\ 0 \end{bmatrix} + 2\begin{bmatrix} 0 \\ 1 \end{bmatrix}$ \\[36pt]
    $w=$ & $\displaystyle 6\begin{bmatrix} 1 \\ 0 \end{bmatrix} - 2 \begin{bmatrix} \ \ 1 \\ -1 \end{bmatrix}$ \\[36pt]
    $w=$ & $\displaystyle \begin{bmatrix} 2 \\ 4 \end{bmatrix} - 2 \begin{bmatrix} -1 \\ \ \ 1 \end{bmatrix}$ \\[36pt]
    $w=$ & $\displaystyle 2\begin{bmatrix} 4 \\ 2 \end{bmatrix} - 4 \begin{bmatrix} 1 \\ 0 \end{bmatrix}
                                              - 2 \begin{bmatrix} 0 \\ 1 \end{bmatrix}$
\end{tabular}
\end{center}

% ----------------------
\end{frame}
%%%%%%%%%%%%%%%%%%%%%%%%%%%%%%%%%%%%%%%%%%%%%%%%%%%%%%%%%%%%%%%%%%%%%%%%
%%%%%%%%%%%%%%%%%%%%%%%%%%%%%%%%%%%%%%%%%%%%%%%%%%%%%%%%%%%%%%%%%%%%%%%%
\begin{frame}
\frametitle{Linear Combinations}

\begin{columns}
\begin{column}{0.35\textwidth}
\textbf{Graphical interpretation:}
\begin{itemize}
    \item   Vector tails can be moved to convenient locations
    \item   Magnitude and direction of vectors is preserved
\end{itemize}
\end{column}
\begin{column}{0.55\textwidth}
\begin{center}
    \pgfimage[height=0.8\textwidth]{./figs/linearCombo2D}
\end{center}
\end{column}
\end{columns}

% ----------------------
\end{frame}
%%%%%%%%%%%%%%%%%%%%%%%%%%%%%%%%%%%%%%%%%%%%%%%%%%%%%%%%%%%%%%%%%%%%%%%%
%%%%%%%%%%%%%%%%%%%%%%%%%%%%%%%%%%%%%%%%%%%%%%%%%%%%%%%%%%%%%%%%%%%%%%%%
\begin{frame}
\frametitle{Vector Inner Product}

In physics, analytical geometry, and engineering, the
\textbf{dot product} has a geometric interpretation
\begin{equation*}
    \sigma = x \cdot y\ \ \ \Longleftrightarrow\ \ \
    \sigma = \sum_{i = 1}^{n}{x_i y_i}
\end{equation*}
\begin{equation*}
    x\cdot y = \|x\|_{2}\, \|y\|_{2} \cos\theta
\end{equation*}

% ----------------------
\end{frame}
%%%%%%%%%%%%%%%%%%%%%%%%%%%%%%%%%%%%%%%%%%%%%%%%%%%%%%%%%%%%%%%%%%%%%%%%
%%%%%%%%%%%%%%%%%%%%%%%%%%%%%%%%%%%%%%%%%%%%%%%%%%%%%%%%%%%%%%%%%%%%%%%%
\begin{frame}
\frametitle{Vector Inner Product}

The inner product of $x$ and $y$ \emph{requires} that
$x$ be a row vector
$y$ be a column vector
\begin{equation*}
    \begin{bmatrix}x_1\ \ x_2\ \ x_3\ \ x_4\end{bmatrix}
    \begin{bmatrix}y_1\\  y_2\\  y_3\\  y_4\end{bmatrix}
    = x_1 y_1 + x_2 y_2 + x_3 y_3 + x_4 y_4
\end{equation*}

% ----------------------
\end{frame}
%%%%%%%%%%%%%%%%%%%%%%%%%%%%%%%%%%%%%%%%%%%%%%%%%%%%%%%%%%%%%%%%%%%%%%%%
%%%%%%%%%%%%%%%%%%%%%%%%%%%%%%%%%%%%%%%%%%%%%%%%%%%%%%%%%%%%%%%%%%%%%%%%
\begin{frame}
\frametitle{Vector Inner Product}

For two $n$-element \emph{column} vectors, $u$ and $v$, the inner product is
\begin{equation*}
    \sigma = u^T v\ \ \ \Longleftrightarrow\ \ \
    \sigma = \sum_{i = 1}^{n}{u_i v_i}
\end{equation*}

The inner product is commutative so that\newline
(for two column vectors)
\begin{equation*}
    u^T v = v^T u
\end{equation*}

% ----------------------
\end{frame}
%%%%%%%%%%%%%%%%%%%%%%%%%%%%%%%%%%%%%%%%%%%%%%%%%%%%%%%%%%%%%%%%%%%%%%%%
%%%%%%%%%%%%%%%%%%%%%%%%%%%%%%%%%%%%%%%%%%%%%%%%%%%%%%%%%%%%%%%%%%%%%%%%
\begin{frame}[fragile]
\frametitle{Computing the Inner Product in Matlab}

The \verb|*| operator performs the inner product
if two vectors are compatible.

\begin{lstlisting}[language=matlab]
>> u = (0:3)';         %  u and v are
>> v = (3:-1:0)';      %  column vectors
>> s = u*v
??? Error using ==> *
Inner matrix dimensions must agree.

>> s = u'*v
s =
     4

>> t = v'*u
t =
     4

>> dot(u,v)
ans =
     4
\end{lstlisting}


% ----------------------
\end{frame}
%%%%%%%%%%%%%%%%%%%%%%%%%%%%%%%%%%%%%%%%%%%%%%%%%%%%%%%%%%%%%%%%%%%%%%%%
%%%%%%%%%%%%%%%%%%%%%%%%%%%%%%%%%%%%%%%%%%%%%%%%%%%%%%%%%%%%%%%%%%%%%%%%
\begin{frame}
\frametitle{Vector Outer Product}

\begin{minipage}[t]{0.45\textwidth}
The inner product results in a scalar.

The \emph{outer product} creates a rank-one matrix:
\begin{equation*}
    A = u v^T\ \ \ \Longleftrightarrow\ \ \
    a_{i,j} = u_i v_j
\end{equation*}
\end{minipage}
\hspace{5ex}
\begin{minipage}[t]{0.45\textwidth}
\begin{example}{Outer product of two 4-element column vectors}
\begin{align*}
    u v^T &= \begin{bmatrix}u_1\\ u_2\\ u_3\\ u_4\end{bmatrix}
             \begin{bmatrix}v_1&  v_2&  v_3& v_4\end{bmatrix}   \\[8pt]
          &= \begin{bmatrix} u_1 v_1 & u_1 v_2 & u_1 v_3 & u_1 v_4 \\
                    u_2 v_1 & u_2 v_2 & u_2 v_3 & u_2 v_4 \\
                    u_3 v_1 & u_3 v_2 & u_3 v_3 & u_3 v_4 \\
                    u_4 v_1 & u_4 v_2 & u_4 v_3 & u_4 v_4
              \end{bmatrix}
\end{align*}
\end{example}
\end{minipage}

% ----------------------
\end{frame}
%%%%%%%%%%%%%%%%%%%%%%%%%%%%%%%%%%%%%%%%%%%%%%%%%%%%%%%%%%%%%%%%%%%%%%%%
%%%%%%%%%%%%%%%%%%%%%%%%%%%%%%%%%%%%%%%%%%%%%%%%%%%%%%%%%%%%%%%%%%%%%%%%
\begin{frame}[fragile]
\frametitle{Computing the Outer Product in Matlab}

The \verb|*| operator performs the outer product
if two vectors are compatible.
\begin{lstlisting}[language=matlab]
u = (0:4)';
v = (4:-1:0)';
A = u*v'
A =
     0     0     0     0     0
     4     3     2     1     0
     8     6     4     2     0
    12     9     6     3     0
    16    12     8     4     0
\end{lstlisting}


% ----------------------
\end{frame}
%%%%%%%%%%%%%%%%%%%%%%%%%%%%%%%%%%%%%%%%%%%%%%%%%%%%%%%%%%%%%%%%%%%%%%%%
%%%%%%%%%%%%%%%%%%%%%%%%%%%%%%%%%%%%%%%%%%%%%%%%%%%%%%%%%%%%%%%%%%%%%%%%
\begin{frame}
\frametitle{Vector Norms}

Compare magnitude of scalars with the \emph{absolute value}
\begin{equation*}
    \bigl| \alpha \bigr|  >  \bigl| \beta \bigr|
\end{equation*}


Compare magnitude of vectors with \emph{norms}
\begin{equation*}
    \norm{x} > \norm{y}
\end{equation*}
There are several ways to compute $||x||$.  In other words
the size of two vectors can be compared with different norms.



% ----------------------
\end{frame}
%%%%%%%%%%%%%%%%%%%%%%%%%%%%%%%%%%%%%%%%%%%%%%%%%%%%%%%%%%%%%%%%%%%%%%%%
%%%%%%%%%%%%%%%%%%%%%%%%%%%%%%%%%%%%%%%%%%%%%%%%%%%%%%%%%%%%%%%%%%%%%%%%
\begin{frame}
\frametitle{Vector Norms}

Consider two element vectors, which lie in a plane
\begin{center}
    \pgfimage[height=3cm]{./figs/2Dvectorsab}
    \qquad
    \pgfimage[height=3cm]{./figs/2Dvectorsac}
\end{center}
Use geometric lengths to represent the magnitudes of the vectors
\begin{equation*}
    \ell_a = \sqrt{4^2 + 2^2} = \sqrt{20},\qquad
    \ell_b = \sqrt{2^2 + 4^2} = \sqrt{20},\qquad
    \ell_c = \sqrt{2^2 + 1^2} = \sqrt{5}
\end{equation*}
%% \begin{align*}
%%     \ell_a &= \sqrt{4^2 + 2^2} = \sqrt{20}\\[6pt]
%%     \ell_b &= \sqrt{2^2 + 4^2} = \sqrt{20}\\[6pt]
%%     \ell_c &= \sqrt{2^2 + 1^2} = \sqrt{5}
%% \end{align*}

We conclude that
\begin{equation*}
    \ell_a = \ell_b
    \quad
    \text{and}
    \quad
    \ell_a > \ell_c
\end{equation*}
or
\begin{equation*}
    \norm{a} = \norm{b}
    \quad
    \text{and}
    \quad
    \norm{a} > \norm{c}
\end{equation*}



% ----------------------
\end{frame}
%%%%%%%%%%%%%%%%%%%%%%%%%%%%%%%%%%%%%%%%%%%%%%%%%%%%%%%%%%%%%%%%%%%%%%%%
%%%%%%%%%%%%%%%%%%%%%%%%%%%%%%%%%%%%%%%%%%%%%%%%%%%%%%%%%%%%%%%%%%%%%%%%
\begin{frame}
\frametitle{The $L_2$ Norm}

The notion of a geometric length for 2D or 3D vectors can be extended
vectors with arbitrary numbers of elements.

The result is called the \emph{Euclidian} or $L_2$ norm:
\begin{equation*}
    \|x\|_{2} = \bigl( x_1^2 + x_2^2 + \ldots + x_n^2 \bigr)^{1/2}
                 = \left( \sum_{i=1}^{n}{x_i^2} \right)^{1/2}
\end{equation*}

The $L_2$ norm can also be expressed in terms of the inner product
\begin{equation*}
    \|x\|_{2} = \sqrt{x \cdot x} = \sqrt{x^T x}
\end{equation*}

% ----------------------
\end{frame}
%%%%%%%%%%%%%%%%%%%%%%%%%%%%%%%%%%%%%%%%%%%%%%%%%%%%%%%%%%%%%%%%%%%%%%%%
%%%%%%%%%%%%%%%%%%%%%%%%%%%%%%%%%%%%%%%%%%%%%%%%%%%%%%%%%%%%%%%%%%%%%%%%
\begin{frame}
\frametitle{$p$-Norms}

For any positive integer $p$
\begin{equation*}
    \|x\|_{p} = \bigl( |x_1|^p + |x_2|^p + \ldots + |x_n|^p \bigr)^{1/p}
\end{equation*}

The $L_1$ norm is sum of absolute values
\begin{equation*}
    \|x\|_{1} = |x_1| + |x_2| + \ldots + |x_n| = \sum_{i=1}^{n}{|x_i|}
\end{equation*}

The $L_\infty$ norm or \emph{max norm} is
\begin{equation*}
    \|x\|_{\infty} = \max \left(|x_1|, |x_2|, \ldots, |x_n| \right)
                      = \underset{i}{\max} \left( |x_i| \right)
\end{equation*}
Although $p$ can be any positive number, $p=1,2,\infty$ are most commonly used.

% ----------------------
\end{frame}
%%%%%%%%%%%%%%%%%%%%%%%%%%%%%%%%%%%%%%%%%%%%%%%%%%%%%%%%%%%%%%%%%%%%%%%%
%%%%%%%%%%%%%%%%%%%%%%%%%%%%%%%%%%%%%%%%%%%%%%%%%%%%%%%%%%%%%%%%%%%%%%%%
\begin{frame}
\frametitle{Application of Norms}

\textbf{Are two vectors (nearly) equal?}\par
Floating point comparison of two scalars with absolute value:
\begin{equation*}
    \frac{\bigl| \alpha - \beta  \bigr|}{\bigl| \alpha \bigr|} < \delta
\end{equation*}
where $\delta$ is a small tolerance.

Comparison of two vectors with norms:
\begin{equation*}
    \frac{\norm{y-z}}{\norm{z}} < \delta
\end{equation*}

% ----------------------
\end{frame}
%%%%%%%%%%%%%%%%%%%%%%%%%%%%%%%%%%%%%%%%%%%%%%%%%%%%%%%%%%%%%%%%%%%%%%%%
%%%%%%%%%%%%%%%%%%%%%%%%%%%%%%%%%%%%%%%%%%%%%%%%%%%%%%%%%%%%%%%%%%%%%%%%
\begin{frame}
\frametitle{Application of Norms}

Notice that
\begin{equation*}
    \frac{\norm{y-z}}{\norm{z}} < \delta
\end{equation*}
is \textbf{not equivalent to}
\begin{equation*}
    \frac{\norm{y}-\norm{z}}{\norm{z}} < \delta.
\end{equation*}

This comparison is important in convergence tests for sequences
of vectors.  
% See Example~7.3 in the textbook.
% WDG - I could not find this example in the 6th edition

% ----------------------
\end{frame}
%%%%%%%%%%%%%%%%%%%%%%%%%%%%%%%%%%%%%%%%%%%%%%%%%%%%%%%%%%%%%%%%%%%%%%%%
%%%%%%%%%%%%%%%%%%%%%%%%%%%%%%%%%%%%%%%%%%%%%%%%%%%%%%%%%%%%%%%%%%%%%%%%
\begin{frame}
\frametitle{Application of Norms}

\textbf{Creating a Unit Vector}

Given $u=[u_1, u_2, \ldots, u_m]^T$, the unit vector in the direction of $u$ is
\begin{equation*}
    \hat{u} = \frac{u}{\|u\|_{2}}
\end{equation*}

\bigskip
Proof:
\begin{equation*}
    \|{\hat{u}}\|_{2} = {\biggl\|\frac{u}{\|u\|_{2}}\biggr\|}_2 = \frac{1}{\|u\|_{2}}\|u\|_{2} = 1
\end{equation*}

\bigskip
The following are \emph{not} unit vectors
\begin{equation*}
    \frac{u}{\|u\|_{1}}
    \qquad
    \frac{u}{\|u\|_{\infty}}
\end{equation*}



% ----------------------
\end{frame}
%%%%%%%%%%%%%%%%%%%%%%%%%%%%%%%%%%%%%%%%%%%%%%%%%%%%%%%%%%%%%%%%%%%%%%%%
%%%%%%%%%%%%%%%%%%%%%%%%%%%%%%%%%%%%%%%%%%%%%%%%%%%%%%%%%%%%%%%%%%%%%%%%
\begin{frame}
\frametitle{Orthogonal Vectors}

From geometric interpretation of the inner product
\begin{equation*}
    u\cdot v = \|u\|_{2}\,\|v\|_{2}\cos\theta
\end{equation*}
\begin{equation*}
    \cos\theta = \frac{u\cdot v}{\|u\|_{2}\,\|v\|_{2}}
               = \frac{u^T v}{\|u\|_{2}\,\|v\|_{2}}
\end{equation*}

Two vectors are orthogonal when $\theta = \pi/2$ or $u\cdot v = 0$.

In other words
\begin{equation*}
    u^T v = 0
\end{equation*}
\emph{if and only if} $u$ and $v$ are \emph{orthogonal}.

% ----------------------
\end{frame}
%%%%%%%%%%%%%%%%%%%%%%%%%%%%%%%%%%%%%%%%%%%%%%%%%%%%%%%%%%%%%%%%%%%%%%%%
%%%%%%%%%%%%%%%%%%%%%%%%%%%%%%%%%%%%%%%%%%%%%%%%%%%%%%%%%%%%%%%%%%%%%%%%
\begin{frame}
\frametitle{Orthonormal Vectors}

\textbf{Orthonormal vectors} are \textbf{unit vectors} that are \emph{orthogonal}.

A \textbf{unit} vector has an $L_2$ norm of one.

The unit vector in the direction of $u$ is
\begin{equation*}
    \hat{u} = \frac{u}{\|u\|_{2}}
\end{equation*}

Since
\begin{equation*}
    \|u\|_{2} = \sqrt{u\cdot u}
\end{equation*}
it follows that $u\cdot u = 1$ if $u$ is a unit vector.

% ----------------------
\end{frame}
%%%%%%%%%%%%%%%%%%%%%%%%%%%%%%%%%%%%%%%%%%%%%%%%%%%%%%%%%%%%%%%%%%%%%%%%
%%%%%%%%%%%%%%%%%%%%%%%%%%%%%%%%%%%%%%%%%%%%%%%%%%%%%%%%%%%%%%%%%%%%%%%%
\begin{frame}
\frametitle{Matrices}

\begin{itemize}
    \item   Columns and Rows of a Matrix are Vectors
    \item   Addition and Subtraction
    \item   Multiplication by a scalar
    \item   Transpose
    \item   Linear Combinations of Vectors
    \item   Matrix--Vector Product
    \item   Matrix--Matrix Product
\end{itemize}

% ----------------------
\end{frame}
%%%%%%%%%%%%%%%%%%%%%%%%%%%%%%%%%%%%%%%%%%%%%%%%%%%%%%%%%%%%%%%%%%%%%%%%
%%%%%%%%%%%%%%%%%%%%%%%%%%%%%%%%%%%%%%%%%%%%%%%%%%%%%%%%%%%%%%%%%%%%%%%%
\begin{frame}[fragile]
\frametitle{Notation}

The matrix $A$ with $m$ rows and $n$ columns looks like:
\begin{equation*}
    A = \begin{bmatrix} a_{11} & a_{12} & \cdots & a_{1n} \\
                        a_{21} & a_{22} &        & a_{2n} \\
                        \vdots &        &        & \vdots \\
                        a_{m1} &        & \cdots & a_{mn} \\
        \end{bmatrix}
\end{equation*}

\begin{equation*}
    a_{ij} = \text{element in \textbf{row} $i$, and \textbf{column} $j$}
\end{equation*}

In Matlab\ we can define a matrix with
\begin{lstlisting}[language=matlab]
>> A = [ ... ; ... ; ... ]
\end{lstlisting}
where semicolons separate lists of row elements.

The $a_{2,3}$ element of the Matlab\ matrix \texttt{A} is \texttt{A(2,3)}.


% ----------------------
\end{frame}
%%%%%%%%%%%%%%%%%%%%%%%%%%%%%%%%%%%%%%%%%%%%%%%%%%%%%%%%%%%%%%%%%%%%%%%%
%%%%%%%%%%%%%%%%%%%%%%%%%%%%%%%%%%%%%%%%%%%%%%%%%%%%%%%%%%%%%%%%%%%%%%%%
\begin{frame}
\frametitle{Matrices Consist of Row and Column Vectors}

\begin{minipage}[t]{0.45\textwidth}
As a collection of column vectors
\begin{equation*}
    A =
    \begin{bmatrix}\left. \begin{matrix} \\ \\ a_{(1)}  \\ \\ \\ \end{matrix}\right|
                   \left. \begin{matrix} \\ \\ a_{(2)}  \\ \\ \\ \end{matrix}\right|
                   \left. \begin{matrix} \\ \\ {\cdots\ } \\ \\ \\ \end{matrix}\right|
                          \begin{matrix} \\ \\ a_{(n)}  \\ \\ \\ \end{matrix}\negthickspace
    \end{bmatrix}
\end{equation*}
\end{minipage}
\hspace{3ex}
\begin{minipage}[t]{0.45\textwidth}
As a collection of row vectors
\begin{equation*}
    A =
    \begin{bmatrix}\begin{matrix} & & a'_{(1)}& & & \end{matrix} \\
                   \dfrac{\makebox[1.2in]{}}{\makebox[1.2in]{}} \\
                   \begin{matrix} & & a'_{(2)}& & & \end{matrix} \\
                   \dfrac{\makebox[1.2in]{}}{\makebox[1.2in]{}} \\
                   \begin{matrix} & & \vdots & & & \end{matrix} \\
                   \dfrac{\makebox[1.2in]{}}{\makebox[1.2in]{}} \\
                   \begin{matrix} & & a'_{(m)}& & & \end{matrix} \\
    \end{bmatrix}
\end{equation*}
A prime is used to designate a row vector on this and the following pages.
\end{minipage}

% ----------------------
\end{frame}
%%%%%%%%%%%%%%%%%%%%%%%%%%%%%%%%%%%%%%%%%%%%%%%%%%%%%%%%%%%%%%%%%%%%%%%%
%%%%%%%%%%%%%%%%%%%%%%%%%%%%%%%%%%%%%%%%%%%%%%%%%%%%%%%%%%%%%%%%%%%%%%%%
\begin{frame}
\frametitle{Preview of the Row and Column View}

    \begin{center}
        \begin{minipage}{1.5in}
            \begin{center}
                Matrix and\\
                vector operations
            \end{center}
        \end{minipage}

        \hspace{0.125in}
        {\Large$\updownarrow$}
        \hspace{0.125in}

        \begin{minipage}{1.5in}
            \begin{center}
                Row and column\\
                operations
            \end{center}
        \end{minipage}

        \hspace{0.125in}
        {\Large$\updownarrow$}
        \hspace{0.125in}

        \begin{minipage}{1.5in}
            \begin{center}
                Element-by-element\\
                operations
            \end{center}
        \end{minipage}
        \end{center}
% ----------------------
\end{frame}
%%%%%%%%%%%%%%%%%%%%%%%%%%%%%%%%%%%%%%%%%%%%%%%%%%%%%%%%%%%%%%%%%%%%%%%%
%%%%%%%%%%%%%%%%%%%%%%%%%%%%%%%%%%%%%%%%%%%%%%%%%%%%%%%%%%%%%%%%%%%%%%%%
\begin{frame}
\frametitle{Matrix Operations}

\begin{itemize}
    \item   Addition and subtraction
    \item   Multiplication by a Scalar
    \item   Matrix Transpose
    \item   Matrix--Vector Multiplication
    \item   Vector--Matrix Multiplication
    \item   Matrix--Matrix Multiplication
\end{itemize}

% ----------------------
\end{frame}
%%%%%%%%%%%%%%%%%%%%%%%%%%%%%%%%%%%%%%%%%%%%%%%%%%%%%%%%%%%%%%%%%%%%%%%%
%%%%%%%%%%%%%%%%%%%%%%%%%%%%%%%%%%%%%%%%%%%%%%%%%%%%%%%%%%%%%%%%%%%%%%%%
\begin{frame}
\frametitle{Matrix Operations}

\textbf{Addition and subtraction}
\begin{equation*}
    C = A + B
\end{equation*}
or
\begin{equation*}
    c_{i,j} = a_{i,j} + b_{i,j}\ \ i = 1,\ldots,m;\ \ j = 1,\ldots, n
\end{equation*}

\vspace{2ex}

\textbf{Multiplication by a Scalar}
\begin{equation*}
    B = \sigma A
\end{equation*}
or
\begin{equation*}
    b_{i,j} = \sigma a_{i,j}\ \ \ i = 1,\ldots,m;\ \ j = 1,\ldots, n
\end{equation*}

\begin{block}{Note}
    Commas in subscripts are necessary when the subscripts are assigned numerical
    values.  For example, $a_{2,3}$ is the row 2, column 3 element of matrix $A$,
    whereas $a_{23}$ is the $23\mathrm{rd}$ element of vector $a$.  When variables
    appear in indices, such as $a_{ij}$ or $a_{i,j}$, the comma is optional
\end{block}


% ----------------------
\end{frame}
%%%%%%%%%%%%%%%%%%%%%%%%%%%%%%%%%%%%%%%%%%%%%%%%%%%%%%%%%%%%%%%%%%%%%%%%
%%%%%%%%%%%%%%%%%%%%%%%%%%%%%%%%%%%%%%%%%%%%%%%%%%%%%%%%%%%%%%%%%%%%%%%%
\begin{frame}[fragile]
\frametitle{Matrix Transpose}

\begin{equation*}
    B = A^T
\end{equation*}
or
\begin{equation*}
    b_{i,j} = a_{j,i}\ \ \ i = 1,\ldots,m;\ \ j = 1,\ldots, n
\end{equation*}

In Matlab\
\begin{lstlisting}[language=matlab]
>> A = [0 0 0; 0 0 0; 1 2 3; 0 0 0]
A =
	 0     0     0
	 0     0     0
	 1     2     3
	 0     0     0

>> B = A'
B =
	 0     0     1     0
	 0     0     2     0
	 0     0     3     0
\end{lstlisting}

% ----------------------
\end{frame}
%%%%%%%%%%%%%%%%%%%%%%%%%%%%%%%%%%%%%%%%%%%%%%%%%%%%%%%%%%%%%%%%%%%%%%%%
%%%%%%%%%%%%%%%%%%%%%%%%%%%%%%%%%%%%%%%%%%%%%%%%%%%%%%%%%%%%%%%%%%%%%%%%
\begin{frame}
\frametitle{Matrix--Vector Product}

\begin{itemize}
    \item   The Column View
        \begin{itemize}
           \item   gives mathematical insight
           \item[]    \vspace{2ex}
        \end{itemize}
    \item   The Row View
        \begin{itemize}
           \item   easy to do by hand
           \item[]    \vspace{2ex}
        \end{itemize}
    \item   The Vector View
        \begin{itemize}
           \item   A square matrix rotates and stretches a vector
           \item[]    \vspace{2ex}
        \end{itemize}
\end{itemize}

% ----------------------
\end{frame}
%%%%%%%%%%%%%%%%%%%%%%%%%%%%%%%%%%%%%%%%%%%%%%%%%%%%%%%%%%%%%%%%%%%%%%%%
%%%%%%%%%%%%%%%%%%%%%%%%%%%%%%%%%%%%%%%%%%%%%%%%%%%%%%%%%%%%%%%%%%%%%%%%
\begin{frame}
\frametitle{Column View of Matrix--Vector Product}

Consider a \textbf{linear combination of a set of column vectors}
$\{a_{(1)}, a_{(2)}, \ldots, a_{(n)}\}$.  Each $a_{(j)}$ has $m$ elements

Let $x_i$ be a set (a vector) of scalar multipliers
\begin{equation*}
    x_1 a_{(1)} + x_2 a_{(2)} +  \ldots + x_n a_{(n)} = b
\end{equation*}
or
\begin{equation*}
    \sum_{j=1}^n a_{(j)}x_j = b
\end{equation*}
Expand the (hidden) row index
\begin{equation*}
      x_1 \begin{bmatrix} a_{11} \\ a_{21} \\ \vdots \\ a_{m1} \end{bmatrix}
    + x_2 \begin{bmatrix} a_{12} \\ a_{22} \\ \vdots \\ a_{m2} \end{bmatrix}
    + \cdots
    + x_n \begin{bmatrix} a_{1n} \\ a_{2n} \\ \vdots \\ a_{mn} \end{bmatrix}
    = \begin{bmatrix} b_1 \\ b_2 \\ \vdots \\ b_m \end{bmatrix}
\end{equation*}

% ----------------------
\end{frame}
%%%%%%%%%%%%%%%%%%%%%%%%%%%%%%%%%%%%%%%%%%%%%%%%%%%%%%%%%%%%%%%%%%%%%%%%
%%%%%%%%%%%%%%%%%%%%%%%%%%%%%%%%%%%%%%%%%%%%%%%%%%%%%%%%%%%%%%%%%%%%%%%%
\begin{frame}
\frametitle{Column View of Matrix--Vector Product}

Form a matrix with the $a_{(j)}$ as columns
\begin{equation*}
    \begin{bmatrix}\left. \begin{matrix} \\ \\ a_{(1)}  \\ \\ \\ \end{matrix}\right|
                   \left. \begin{matrix} \\ \\ a_{(2)}  \\ \\ \\ \end{matrix}\right|
                   \left. \begin{matrix} \\ \\ {\cdots\ } \\ \\ \\ \end{matrix}\right|
                          \begin{matrix} \\ \\ a_{(n)}  \\ \\ \\ \end{matrix}\negthickspace
    \end{bmatrix}
    \begin{bmatrix} x_1 \\ x_2 \\ \vdots \\ x_n \end{bmatrix}
    =
    \begin{bmatrix}  \\  \\ b \\  \\  \\ \end{bmatrix}
\end{equation*}
Or, writing out the elements
\begin{equation*}
    \begin{bmatrix} a_{11} & a_{12} & \cdots & a_{1n} \\
                    a_{21} & a_{22} & \cdots & a_{2n} \\
                           &        &        &        \\
                    \vdots & \vdots &        & \vdots \\
                           &        &        &        \\
                    a_{m1} & a_{m2} & \cdots & a_{mn}
    \end{bmatrix}
    \begin{bmatrix} x_1 \\ x_2 \\ \vdots \\ x_n \end{bmatrix}
    =
    \begin{bmatrix} b_1 \\ b_2 \\  \\ \vdots \\ \\ b_m\end{bmatrix}
\end{equation*}

% ----------------------
\end{frame}
%%%%%%%%%%%%%%%%%%%%%%%%%%%%%%%%%%%%%%%%%%%%%%%%%%%%%%%%%%%%%%%%%%%%%%%%
%%%%%%%%%%%%%%%%%%%%%%%%%%%%%%%%%%%%%%%%%%%%%%%%%%%%%%%%%%%%%%%%%%%%%%%%
\begin{frame}
\frametitle{Column View of Matrix--Vector Product}

Thus, the matrix-vector product is
\begin{equation*}
    \begin{bmatrix} a_{11} & a_{12} & \cdots & a_{1n} \\
                    a_{21} & a_{22} & \cdots & a_{2n} \\
                           &        &        &        \\
                    \vdots & \vdots &        & \vdots \\
                           &        &        &        \\
                    a_{m1} & a_{m2} & \cdots & a_{mn}
    \end{bmatrix}
    \begin{bmatrix} x_1 \\ x_2 \\ \vdots \\ x_n \end{bmatrix}
    =
    \begin{bmatrix} b_1 \\ b_2 \\  \\ \vdots \\ \\ b_m\end{bmatrix}
\end{equation*}

Save space with matrix notation
\begin{equation*}
    Ax = b
\end{equation*}

% ----------------------
\end{frame}
%%%%%%%%%%%%%%%%%%%%%%%%%%%%%%%%%%%%%%%%%%%%%%%%%%%%%%%%%%%%%%%%%%%%%%%%
%%%%%%%%%%%%%%%%%%%%%%%%%%%%%%%%%%%%%%%%%%%%%%%%%%%%%%%%%%%%%%%%%%%%%%%%
\begin{frame}
\frametitle{Column View of Matrix--Vector Product}

\begin{center}
\begin{minipage}{4.75in}
    \bfseries
    The matrix--vector product $b = Ax$ \\
    produces a vector $b$ from a linear \\
    combination of the columns in $A$.
\end{minipage}
\end{center}

\begin{equation*}
    b = A x\ \ \ \Longleftrightarrow
           \ \ \ b_i = \sum_{j=1}^{n}{ a_{ij} x_j }
\end{equation*}
where $x$ and $b$ are column vectors


% ----------------------
\end{frame}
%%%%%%%%%%%%%%%%%%%%%%%%%%%%%%%%%%%%%%%%%%%%%%%%%%%%%%%%%%%%%%%%%%%%%%%%
%%%%%%%%%%%%%%%%%%%%%%%%%%%%%%%%%%%%%%%%%%%%%%%%%%%%%%%%%%%%%%%%%%%%%%%%
\begin{frame}[fragile]
\frametitle{Column View of Matrix--Vector Product}

\begin{lstlisting}[mathescape,caption=Matrix--Vector Multiplication by Columns,label=algo:MatVecCol]
  initialize:  $b = \mathtt{zeros}(m,1)$       
  for $j=1,\ldots,n$      
    for $i=1,\ldots,m$        
      $b(i) = A(i,j) x(j) + b(i)$  
    end                     
  end
\end{lstlisting}

% ----------------------
\end{frame}
%%%%%%%%%%%%%%%%%%%%%%%%%%%%%%%%%%%%%%%%%%%%%%%%%%%%%%%%%%%%%%%%%%%%%%%%
%%%%%%%%%%%%%%%%%%%%%%%%%%%%%%%%%%%%%%%%%%%%%%%%%%%%%%%%%%%%%%%%%%%%%%%%
\begin{frame}
\frametitle{Compatibility Requirement}

\textbf{Inner dimensions must agree}
\begin{equation*}
    \begin{array}{cccc}
                A       &        x        & = & b \\[3pt]
        [\matdim{m}{n}] & [\matdim{n}{1}] & = & [\matdim{m}{1}]
    \end{array}
\end{equation*}


% ----------------------
\end{frame}
%%%%%%%%%%%%%%%%%%%%%%%%%%%%%%%%%%%%%%%%%%%%%%%%%%%%%%%%%%%%%%%%%%%%%%%%
%%%%%%%%%%%%%%%%%%%%%%%%%%%%%%%%%%%%%%%%%%%%%%%%%%%%%%%%%%%%%%%%%%%%%%%%
\begin{frame}
\frametitle{Row View of Matrix--Vector Product}

Consider the following matrix--vector product written out as a linear
combination of matrix columns
\begin{equation*}
    \left[\negthickspace
          \begin{array}{rrrr}  5 &  0  &  0  &  -1  \\
                              -3 &  4  & -7  &   1  \\
                               1 &  2  &  3  &   6  \end{array}\negthickspace\right]
    \left[\negthickspace
          \begin{array}{r}  4 \\  2  \\  -3  \\  -1 \end{array}\negthickspace\right]
\end{equation*}
\begin{equation*}
    \ \ \ \ \ \ \ \ \ \ \ \ \ \
    = 4\left[\negthickspace\begin{array}{r}  5 \\ -3 \\  1 \end{array}\negthickspace\right]
    + 2\left[\negthickspace\begin{array}{r}  0 \\  4 \\  2 \end{array}\negthickspace\right]
    - 3\left[\negthickspace\begin{array}{r}  0 \\ -7 \\  3 \end{array}\negthickspace\right]
    - 1\left[\negthickspace\begin{array}{r} -1 \\  1 \\  6 \end{array}\negthickspace\right]
\end{equation*}

\vspace{2ex}
This is the column view.

% ----------------------
\end{frame}
%%%%%%%%%%%%%%%%%%%%%%%%%%%%%%%%%%%%%%%%%%%%%%%%%%%%%%%%%%%%%%%%%%%%%%%%
%%%%%%%%%%%%%%%%%%%%%%%%%%%%%%%%%%%%%%%%%%%%%%%%%%%%%%%%%%%%%%%%%%%%%%%%
\begin{frame}
\frametitle{Row View of Matrix--Vector Product}

Now, group the multiplication and addition operations by row:
\begin{equation*}
      4\left[\negthickspace\begin{array}{r}  5 \\ -3 \\  1 \end{array}\negthickspace\right]
    + 2\left[\negthickspace\begin{array}{r}  0 \\  4 \\  2 \end{array}\negthickspace\right]
    - 3\left[\negthickspace\begin{array}{r}  0 \\ -7 \\  3 \end{array}\negthickspace\right]
    - 1\left[\negthickspace\begin{array}{r} -1 \\  1 \\  6 \end{array}\negthickspace\right]
    \qquad\qquad\qquad\qquad
\end{equation*}
\begin{equation*}
    \hfill
    = \left[\negthickspace
          \begin{array}{rcrcrcr}
                     (5)(4)  & + & (0)(2) & + & (0)(-3)  & + & (-1)(-1) \\
                     (-3)(4) & + & (4)(2) & + & (-7)(-3) & + & (1)(-1)  \\
                     (1)(4)  & + & (2)(2) & + & (3)(-3)  & + & (6)(-1)  \end{array}\negthickspace\right]
    \quad=\quad
    \left[\negthickspace
          \begin{array}{r}  21 \\  16  \\ -7 \end{array}\negthickspace\right]
\end{equation*}

\vspace{3ex}
Final result is identical to that obtained with the column view.

% ----------------------
\end{frame}
%%%%%%%%%%%%%%%%%%%%%%%%%%%%%%%%%%%%%%%%%%%%%%%%%%%%%%%%%%%%%%%%%%%%%%%%
%%%%%%%%%%%%%%%%%%%%%%%%%%%%%%%%%%%%%%%%%%%%%%%%%%%%%%%%%%%%%%%%%%%%%%%%
\begin{frame}
\frametitle{Row View of Matrix--Vector Product}

Product of a \matdim{3}{4} matrix, $A$, with a \matdim{4}{1} vector, $x$, looks like
\begin{equation*}
    \begin{bmatrix}\begin{matrix} & & a'_{(1)}& & & \end{matrix} \\
                   \dfrac{\makebox[1.2in]{}}{\makebox[1.2in]{}} \\
                   \begin{matrix} & & a'_{(2)}& & & \end{matrix} \\
                   \dfrac{\makebox[1.2in]{}}{\makebox[1.2in]{}} \\
                   \begin{matrix} & & a'_{(3)}& & & \end{matrix}
    \end{bmatrix}
    \begin{bmatrix} x_1 \\ x_2 \\ x_3 \\ x_4 \end{bmatrix}
    =
    \begin{bmatrix} a'_{(1)}\cdot x \\[4pt] a'_{(2)}\cdot x \\[4pt] a'_{(3)}\cdot x \end{bmatrix}
    =
    \begin{bmatrix} b_1 \\ b_2 \\ b_3 \end{bmatrix}
\end{equation*}
where $a'_{(1)}$, $a'_{(2)}$, and $a'_{(3)}$, are the \emph{row vectors}
constituting the $A$ matrix.

\vspace{3ex}

\begin{center}
    \begin{minipage}{4.75in}
        \bfseries
        The matrix--vector product $b = Ax$\\
        produces elements in $b$ by forming\\
        inner products of the rows of $A$ with $x$.
    \end{minipage}
\end{center}
% ----------------------
\end{frame}
%%%%%%%%%%%%%%%%%%%%%%%%%%%%%%%%%%%%%%%%%%%%%%%%%%%%%%%%%%%%%%%%%%%%%%%%
%%%%%%%%%%%%%%%%%%%%%%%%%%%%%%%%%%%%%%%%%%%%%%%%%%%%%%%%%%%%%%%%%%%%%%%%
\begin{frame}
\frametitle{Row View of Matrix--Vector Product}

\begin{center}
    \pgfimage[height=3cm]{./figs/matrixVectorRows}
\end{center}

% ----------------------
\end{frame}
%%%%%%%%%%%%%%%%%%%%%%%%%%%%%%%%%%%%%%%%%%%%%%%%%%%%%%%%%%%%%%%%%%%%%%%%
%%%%%%%%%%%%%%%%%%%%%%%%%%%%%%%%%%%%%%%%%%%%%%%%%%%%%%%%%%%%%%%%%%%%%%%%
\begin{frame}
\frametitle{Vector View of Matrix--Vector Product}

If $A$ is square, the product $Ax$ has the effect of
stretching and rotating $x$.

Pure stretching of the column vector
\begin{equation*}
    \begin{bmatrix}2 & 0 & 0\\ 0 & 2 & 0 \\ 0 & 0 & 2 \end{bmatrix}
    \begin{bmatrix}1 \\ 2 \\ 3 \end{bmatrix}
    =
    \begin{bmatrix}2 \\ 4 \\ 6 \end{bmatrix}
\end{equation*}

Pure rotation of the column vector
\begin{equation*}
    \begin{bmatrix}0 & -1 & 0\\ 1 & 0 & 0 \\ 0 & 0 & 1 \end{bmatrix}
    \begin{bmatrix}1 \\ 0 \\ 0 \end{bmatrix}
    =
    \begin{bmatrix}0 \\ 1 \\ 0 \end{bmatrix}
\end{equation*}

% ----------------------
\end{frame}
%%%%%%%%%%%%%%%%%%%%%%%%%%%%%%%%%%%%%%%%%%%%%%%%%%%%%%%%%%%%%%%%%%%%%%%%
%%%%%%%%%%%%%%%%%%%%%%%%%%%%%%%%%%%%%%%%%%%%%%%%%%%%%%%%%%%%%%%%%%%%%%%%
\begin{frame}
\frametitle{Vector--Matrix Product}

\textbf{Matrix--vector product}
\begin{center}
    \pgfimage[height=3cm]{./figs/matrixVectorProduct}
\end{center}

\textbf{Vector--Matrix product}
\begin{center}
    \pgfimage[height=3cm]{./figs/vectorMatrixProduct}
\end{center}

% ----------------------
\end{frame}
%%%%%%%%%%%%%%%%%%%%%%%%%%%%%%%%%%%%%%%%%%%%%%%%%%%%%%%%%%%%%%%%%%%%%%%%
%%%%%%%%%%%%%%%%%%%%%%%%%%%%%%%%%%%%%%%%%%%%%%%%%%%%%%%%%%%%%%%%%%%%%%%%
\begin{frame}
\frametitle{Vector--Matrix Product}

\textbf{Compatibility Requirement:}
\textbf{Inner dimensions must agree}
\begin{equation*}
    \begin{array}{cccc}
                u       &        A        & = & v \\[3pt]
        [\matdim{1}{m}] & [\matdim{m}{n}] & = & [\matdim{1}{n}]
    \end{array}
\end{equation*}
%Inner dimensions must agree

% ----------------------
\end{frame}
%%%%%%%%%%%%%%%%%%%%%%%%%%%%%%%%%%%%%%%%%%%%%%%%%%%%%%%%%%%%%%%%%%%%%%%%
%%%%%%%%%%%%%%%%%%%%%%%%%%%%%%%%%%%%%%%%%%%%%%%%%%%%%%%%%%%%%%%%%%%%%%%%
\begin{frame}
\frametitle{Matrix--Matrix Product}

Computations can be organized in \textbf{six different ways}
We'll focus on just two
\begin{itemize}
    \item   Column View --- extension of column view of matrix--vector product
    \item   Row View --- inner product algorithm, extension of column view of matrix--vector product
\end{itemize}

% ----------------------
\end{frame}
%%%%%%%%%%%%%%%%%%%%%%%%%%%%%%%%%%%%%%%%%%%%%%%%%%%%%%%%%%%%%%%%%%%%%%%%
%%%%%%%%%%%%%%%%%%%%%%%%%%%%%%%%%%%%%%%%%%%%%%%%%%%%%%%%%%%%%%%%%%%%%%%%
\begin{frame}
\frametitle{Column View of Matrix--Matrix Product}

The product $AB$ produces a matrix $C$.  The columns of $C$
are linear combinations of the columns of $A$.
\begin{equation*}
    AB = C \qquad \Longleftrightarrow \qquad c_{(j)} = Ab_{(j)}
\end{equation*}
$c_{(j)}$ and $b_{(j)}$ are column vectors.

\begin{center}
    \pgfimage[height=3cm]{./figs/matrixMatrixColumns}
\end{center}

The column view of the matrix--matrix product $AB=C$ is helpful because
it shows the relationship between the columns of $A$ and the columns of $C$.

% ----------------------
\end{frame}
%%%%%%%%%%%%%%%%%%%%%%%%%%%%%%%%%%%%%%%%%%%%%%%%%%%%%%%%%%%%%%%%%%%%%%%%
%%%%%%%%%%%%%%%%%%%%%%%%%%%%%%%%%%%%%%%%%%%%%%%%%%%%%%%%%%%%%%%%%%%%%%%%
\begin{frame}
\frametitle{Inner Product (Row) View of Matrix--Matrix Product}

The product $AB$ produces a matrix $C$.  The $c_{ij}$ element
is the \emph{inner product} of row $i$ of $A$ and column $j$ of $B$.
\begin{equation*}
    AB = C \qquad \Longleftrightarrow \qquad c_{ij} = a'_{(i)}b_{(j)}
\end{equation*}
$a'_{(i)}$ is a row vector, $b_{(j)}$ is a column vector.

\begin{center}
    \pgfimage[height=3cm]{./figs/matrixMatrixRows}
\end{center}

The inner product view of the matrix--matrix product is easier
to use for hand calculations.


% ----------------------
\end{frame}
%%%%%%%%%%%%%%%%%%%%%%%%%%%%%%%%%%%%%%%%%%%%%%%%%%%%%%%%%%%%%%%%%%%%%%%%
%%%%%%%%%%%%%%%%%%%%%%%%%%%%%%%%%%%%%%%%%%%%%%%%%%%%%%%%%%%%%%%%%%%%%%%%
\begin{frame}
\frametitle{Matrix--Matrix Product Summary}


The \textbf{Matrix--vector product} looks like:
\begin{equation*}
    \begin{bmatrix}\bullet & \bullet & \bullet \\
                    \bullet & \bullet & \bullet \\
                    \bullet & \bullet & \bullet \\
                    \bullet & \bullet & \bullet \end{bmatrix}
    \begin{bmatrix}\bullet \\ \bullet \\ \bullet \end{bmatrix}
    =
    \begin{bmatrix}\bullet \\ \bullet \\ \bullet \\ \bullet \end{bmatrix}
\end{equation*}

The \textbf{vector--Matrix product} looks like:
\begin{equation*}
    \begin{bmatrix}\bullet & \bullet & \bullet & \bullet \end{bmatrix}
    \begin{bmatrix}\bullet & \bullet & \bullet \\
                    \bullet & \bullet & \bullet \\
                    \bullet & \bullet & \bullet \\
                    \bullet & \bullet & \bullet \end{bmatrix}
    =
    \begin{bmatrix}\bullet & \bullet & \bullet \end{bmatrix}
\end{equation*}

% ----------------------
\end{frame}
%%%%%%%%%%%%%%%%%%%%%%%%%%%%%%%%%%%%%%%%%%%%%%%%%%%%%%%%%%%%%%%%%%%%%%%%
%%%%%%%%%%%%%%%%%%%%%%%%%%%%%%%%%%%%%%%%%%%%%%%%%%%%%%%%%%%%%%%%%%%%%%%%
\begin{frame}
\frametitle{Matrix--Matrix Product Summary}

The \textbf{Matrix--Matrix product} looks like:
\begin{equation*}
    \begin{bmatrix}\bullet & \bullet & \bullet \\
                    \bullet & \bullet & \bullet \\
                    \bullet & \bullet & \bullet \\
                    \bullet & \bullet & \bullet \end{bmatrix}
    \begin{bmatrix}\bullet & \bullet & \bullet & \bullet \\
                    \bullet & \bullet & \bullet & \bullet \\
                    \bullet & \bullet & \bullet & \bullet \end{bmatrix}
    =
    \begin{bmatrix}\bullet & \bullet & \bullet & \bullet  \\
                    \bullet & \bullet & \bullet & \bullet \\
                    \bullet & \bullet & \bullet & \bullet \\
                    \bullet & \bullet & \bullet & \bullet \end{bmatrix}
\end{equation*}


% ----------------------
\end{frame}
%%%%%%%%%%%%%%%%%%%%%%%%%%%%%%%%%%%%%%%%%%%%%%%%%%%%%%%%%%%%%%%%%%%%%%%%
%%%%%%%%%%%%%%%%%%%%%%%%%%%%%%%%%%%%%%%%%%%%%%%%%%%%%%%%%%%%%%%%%%%%%%%%
\begin{frame}
\frametitle{Matrix--Matrix Product Summary}


\textbf{Compatibility Requirement}
\begin{equation*}
    \begin{array}{cccc}
                A       &        B        & = & C \\[3pt]
        [\matdim{m}{r}] & [\matdim{r}{n}] & = & [\matdim{m}{n}]
    \end{array}
\end{equation*}
Inner dimensions must agree


Also, in general
\begin{equation*}
    AB \neq BA
\end{equation*}

% ----------------------
\end{frame}
%%%%%%%%%%%%%%%%%%%%%%%%%%%%%%%%%%%%%%%%%%%%%%%%%%%%%%%%%%%%%%%%%%%%%%%%
%%%%%%%%%%%%%%%%%%%%%%%%%%%%%%%%%%%%%%%%%%%%%%%%%%%%%%%%%%%%%%%%%%%%%%%%
\begin{frame}
\frametitle{Mathematical Properties of Vectors and Matrices}


\begin{itemize}
    \item   Linear Independence
    \item   Vector Spaces
    \item   Subspaces associated with matrices
    \item   Matrix Rank
\end{itemize}

% ----------------------
\end{frame}
%%%%%%%%%%%%%%%%%%%%%%%%%%%%%%%%%%%%%%%%%%%%%%%%%%%%%%%%%%%%%%%%%%%%%%%%
%%%%%%%%%%%%%%%%%%%%%%%%%%%%%%%%%%%%%%%%%%%%%%%%%%%%%%%%%%%%%%%%%%%%%%%%
\begin{frame}
\frametitle{Linear Independence}

Two vectors lying along the same line are not independent
\begin{equation*}
    u = \begin{bmatrix}1 \\ 1 \\ 1 \end{bmatrix}
    \qquad
    \text{and}
    \qquad
    v = -2u = \begin{bmatrix}-2 \\ -2 \\ -2 \end{bmatrix}
\end{equation*}

Any two independent vectors, for example,
\begin{equation*}
    v = \begin{bmatrix}-2 \\ -2 \\ -2 \end{bmatrix}
    \qquad
    \text{and}
    \qquad
    w = \begin{bmatrix}0 \\ 0 \\ 1 \end{bmatrix}
\end{equation*}
define a plane. Any other vector in this plane of $v$ and $w$ can be
represented by
\begin{equation*}
    x = \alpha v + \beta w
\end{equation*}
$x$ is \textbf{linearly dependent} on $v$ and $w$ because it
can be formed by a linear combination of $v$ and $w$.

% ----------------------
\end{frame}
%%%%%%%%%%%%%%%%%%%%%%%%%%%%%%%%%%%%%%%%%%%%%%%%%%%%%%%%%%%%%%%%%%%%%%%%
%%%%%%%%%%%%%%%%%%%%%%%%%%%%%%%%%%%%%%%%%%%%%%%%%%%%%%%%%%%%%%%%%%%%%%%%
\begin{frame}
\frametitle{Linear Independence}

A set of vectors is linearly independent if it is impossible to use a
linear combination of vectors in the set to create another vector in the set.

Linear independence is easy to see for vectors that are orthogonal,
for example,
\begin{equation*}
    \left[
        \negthickspace\begin{array}{r}4 \\ 0 \\  0\\ 0\end{array}\negthickspace
    \right],
    \ \ \ \ \ \ \ \ \
    \left[
        \negthickspace\begin{array}{r}0 \\ -3 \\  0\\ 0\end{array}\negthickspace
    \right],
    \ \ \ \ \ \ \ \ \
    \left[
        \negthickspace\begin{array}{r}0 \\ 0 \\  1\\ 0\end{array}\negthickspace
    \right]
\end{equation*}
are linearly independent.

% ----------------------
\end{frame}
%%%%%%%%%%%%%%%%%%%%%%%%%%%%%%%%%%%%%%%%%%%%%%%%%%%%%%%%%%%%%%%%%%%%%%%%
%%%%%%%%%%%%%%%%%%%%%%%%%%%%%%%%%%%%%%%%%%%%%%%%%%%%%%%%%%%%%%%%%%%%%%%%
\begin{frame}
\frametitle{Linear Independence}

Consider two linearly independent vectors, $u$ and $v$.

If a third vector, $w$, \emph{cannot} be expressed as a linear
combination of $u$ and $v$, then the set $\{u,v,w\}$ is
linearly independent.

In other words, if $\{u,v,w\}$ is linearly independent then
\begin{equation*}
    \alpha u + \beta v = \delta w
\end{equation*}
can be true \emph{only if} $\alpha = \beta = \delta = 0$.


More generally, if the only solution to
\begin{equation}    \label{eq:linDep}
    \alpha_1 v_{(1)} + \alpha_2 v_{(2)} + \cdots + \alpha_n v_{(n)} = 0
\end{equation}
is $\alpha_1 = \alpha_2 = \ldots = \alpha_n = 0$, then the set
$\{v_{(1)},v_{(2)},\ldots,v_{(n)}\}$
is \textbf{linearly independent}.
Conversely, if equation~\eqref{eq:linDep} is satisfied by at least
one nonzero $\alpha_i$, then the set of vectors is \textbf{linearly dependent}.

% ----------------------
\end{frame}
%%%%%%%%%%%%%%%%%%%%%%%%%%%%%%%%%%%%%%%%%%%%%%%%%%%%%%%%%%%%%%%%%%%%%%%%
%%%%%%%%%%%%%%%%%%%%%%%%%%%%%%%%%%%%%%%%%%%%%%%%%%%%%%%%%%%%%%%%%%%%%%%%
\begin{frame}
\frametitle{Linear Independence}

Let the set of vectors $\{v_{(1)},v_{(2)},\ldots,v_{(n)}\}$
be organized as the columns of a matrix.  Then the condition
of linear independence is
\begin{equation}      \label{eq:linDepMat}
    \begin{bmatrix}\left. \begin{matrix} \\ \\ v_{(1)}  \\ \\ \\ \end{matrix}\right|
                   \left. \begin{matrix} \\ \\ v_{(2)}  \\ \\ \\ \end{matrix}\right|
                   \left. \begin{matrix} \\ \\ {\cdots\ } \\ \\ \\ \end{matrix}\right|
                          \begin{matrix} \\ \\ v_{(n)}  \\ \\ \\ \end{matrix}\negthickspace
    \end{bmatrix}
    \begin{bmatrix} \alpha_1 \\ \alpha_2 \\ \vdots \\ \alpha_n \end{bmatrix}
    =
    \begin{bmatrix}  0 \\  0 \\ \vdots \\ 0 \end{bmatrix}
\end{equation}

\begin{center}
    \begin{minipage}{4.5in}
        \bfseries
        The columns of the \matdim{m}{n} matrix, $A$, are linearly
        independent if and only if $x=(0,0,\ldots,0)^T$ is the only $n$
        element column vector that satisfies $Ax=0$.
    \end{minipage}
\end{center}


% ----------------------
\end{frame}
%%%%%%%%%%%%%%%%%%%%%%%%%%%%%%%%%%%%%%%%%%%%%%%%%%%%%%%%%%%%%%%%%%%%%%%%
%%%%%%%%%%%%%%%%%%%%%%%%%%%%%%%%%%%%%%%%%%%%%%%%%%%%%%%%%%%%%%%%%%%%%%%%
\begin{frame}
\frametitle{Vector Spaces}

\begin{itemize}
    \item   Spaces and Subspaces
    \item   Basis of a Subspace
    \item   Subspaces associated with Matrices
\end{itemize}

% ----------------------
\end{frame}
%%%%%%%%%%%%%%%%%%%%%%%%%%%%%%%%%%%%%%%%%%%%%%%%%%%%%%%%%%%%%%%%%%%%%%%%
%%%%%%%%%%%%%%%%%%%%%%%%%%%%%%%%%%%%%%%%%%%%%%%%%%%%%%%%%%%%%%%%%%%%%%%%
\begin{frame}
\frametitle{Spaces and Subspaces}

Group vectors according to number of elements they have.
Vectors from these different groups cannot be mixed.

\begin{center}
    \renewcommand{\arraystretch}{1.3}
    \begin{align*}
        \mathbf{R}^1 & =  \text{Space of all vectors with one element.}\\
                     &    \text{These vectors define the points along a line.}\\
        \mathbf{R}^2 & =  \text{Space of all vectors with two elements.}\\
                     &    \text{These vectors define the points in a plane.} \\
        \mathbf{R}^n & =  \text{Space of all vectors with $n$ elements.}\\
                     &    \text{These vectors define the points in an}\\
                     &    \text{$n$-dimensional space (hyperplane).}\\
    \end{align*}
\end{center}

% ----------------------
\end{frame}
%%%%%%%%%%%%%%%%%%%%%%%%%%%%%%%%%%%%%%%%%%%%%%%%%%%%%%%%%%%%%%%%%%%%%%%%
%%%%%%%%%%%%%%%%%%%%%%%%%%%%%%%%%%%%%%%%%%%%%%%%%%%%%%%%%%%%%%%%%%%%%%%%
\begin{frame}
\frametitle{Subspaces}

\begin{columns}
\begin{column}{0.5\textwidth}
The three vectors
\begin{equation*}
    u = \left[\negthickspace\begin{array}{r}  1 \\  2 \\  0 \end{array}\negthickspace\right],
    \quad
    v = \left[\negthickspace\begin{array}{r} -2 \\  1 \\ 3 \end{array}\negthickspace\right],
    \quad
    w = \left[\negthickspace\begin{array}{r}  3 \\  1 \\ -3 \end{array}\negthickspace\right],
\end{equation*}
lie in the same plane.  The vectors have three elements each, so they
belong to $\mathbf{R}^3$, but they \textbf{span} a \textbf{subspace} of
$\mathbf{R}^3$.
\end{column}
\begin{column}{0.45\textwidth}
    \pgfimage[height=5cm]{./figs/span2in3}
\end{column}
\end{columns}

% ----------------------
\end{frame}
%%%%%%%%%%%%%%%%%%%%%%%%%%%%%%%%%%%%%%%%%%%%%%%%%%%%%%%%%%%%%%%%%%%%%%%%
%%%%%%%%%%%%%%%%%%%%%%%%%%%%%%%%%%%%%%%%%%%%%%%%%%%%%%%%%%%%%%%%%%%%%%%%
\begin{frame}
\frametitle{Basis and Dimension of a Subspace}

\begin{itemize}
    \item   A \textbf{basis} for a subspace is a set of \textbf{linearly independent}
            vectors that \textbf{span} the subspace.

    \item   Since a basis set must be linearly independent, it also must have the
            smallest number of vectors necessary to span the space.  (Each vector makes a
            unique contribution to spanning some other direction in the space.)

    \item   The number of vectors in a basis set is equal to the \textbf{dimension}
            of the \textbf{subspace} that these vectors span.

    \item   Mutually orthogonal vectors (an orthogonal set) form convenient basis
            sets, but basis sets need not be orthogonal.
\end{itemize}

% ----------------------
\end{frame}
%%%%%%%%%%%%%%%%%%%%%%%%%%%%%%%%%%%%%%%%%%%%%%%%%%%%%%%%%%%%%%%%%%%%%%%%
%%%%%%%%%%%%%%%%%%%%%%%%%%%%%%%%%%%%%%%%%%%%%%%%%%%%%%%%%%%%%%%%%%%%%%%%
\begin{frame}
\frametitle{Subspaces Associated with Matrices}

The matrix--vector product
\begin{equation*}
    y = Ax
\end{equation*}
creates $y$ from a linear combination of the columns of $A$

The column vectors of $A$ form a basis for the \textbf{column space}
or \textbf{range} of $A$.

% ----------------------
\end{frame}
%%%%%%%%%%%%%%%%%%%%%%%%%%%%%%%%%%%%%%%%%%%%%%%%%%%%%%%%%%%%%%%%%%%%%%%%
%%%%%%%%%%%%%%%%%%%%%%%%%%%%%%%%%%%%%%%%%%%%%%%%%%%%%%%%%%%%%%%%%%%%%%%%
\begin{frame}
\frametitle{Matrix Rank}

\begin{itemize}
  \item The \textbf{rank} of a matrix, $A$, is the number of linearly independent
columns in $A$.

  \item \rank{A} is the dimension of the column space of $A$.

  \item Numerical computation of \rank{A} is tricky due to roundoff.
\end{itemize}

Consider
\begin{equation*}
    u = \begin{bmatrix} 1 \\ 0 \\ 
    \only<1>{0}
    \only<2>{0.00001}
    \only<3>{\epsm}
    \end{bmatrix}
    \ \ \ \ \ \ \ \
    v = \begin{bmatrix} 0 \\ 1 \\ 0   \end{bmatrix}
    \ \ \ \ \ \ \ \
    w = \begin{bmatrix} 1 \\ 1 \\ 0 \end{bmatrix}
\end{equation*}

\vspace{1ex}
Do these vectors span $\mathbf{R}^3$?
% ----------------------
\end{frame}
%%%%%%%%%%%%%%%%%%%%%%%%%%%%%%%%%%%%%%%%%%%%%%%%%%%%%%%%%%%%%%%%%%%%%%%%
%%%%%%%%%%%%%%%%%%%%%%%%%%%%%%%%%%%%%%%%%%%%%%%%%%%%%%%%%%%%%%%%%%%%%%%%
\begin{frame}[fragile]
\frametitle{Matrix Rank (2)}

We can use Matlab's built-in \cmd{rank} function
for exploratory calculations on (relatively) small matrices
\begin{lstlisting}[language=matlab]
>> A = [1 0 0; 0 1 0; 0 0 1e-5]    % A(3,3) is small
A =
    1.0000         0         0
         0    1.0000         0
         0         0    0.0000

>> rank(A)
ans =
     3
\end{lstlisting}

% ----------------------
\end{frame}
%%%%%%%%%%%%%%%%%%%%%%%%%%%%%%%%%%%%%%%%%%%%%%%%%%%%%%%%%%%%%%%%%%%%%%%%
%%%%%%%%%%%%%%%%%%%%%%%%%%%%%%%%%%%%%%%%%%%%%%%%%%%%%%%%%%%%%%%%%%%%%%%%
\begin{frame}[fragile]
\frametitle{Matrix Rank (2)}

Repeat numerical calculation of rank with smaller diagonal entry
\begin{lstlisting}[language=matlab]
>> A(3,3) = eps/2    % A(3,3) is even smaller
A =
    1.0000         0         0
         0    1.0000         0
         0         0    0.0000

>> rank(A)
ans =
     2
\end{lstlisting}
Even though \texttt{A(3,3)} is not identically zero, it is
small enough that the matrix is \emph{numerically} rank-deficient

% ----------------------
\end{frame}
%%%%%%%%%%%%%%%%%%%%%%%%%%%%%%%%%%%%%%%%%%%%%%%%%%%%%%%%%%%%%%%%%%%%%%%%
%%%%%%%%%%%%%%%%%%%%%%%%%%%%%%%%%%%%%%%%%%%%%%%%%%%%%%%%%%%%%%%%%%%%%%%%
\begin{frame}
\frametitle{Special Matrices}

\begin{itemize}
    \item   Diagonal Matrices
    \item   Tridiagonal Matrices
    \item   The Identity Matrix
    \item   The Matrix Inverse
    \item   Symmetric Matrices
    \item   Positive Definite Matrices
    \item   Orthogonal Matrices
    \item   Permutation Matrices
\end{itemize}

% ----------------------
\end{frame}
%%%%%%%%%%%%%%%%%%%%%%%%%%%%%%%%%%%%%%%%%%%%%%%%%%%%%%%%%%%%%%%%%%%%%%%%
%%%%%%%%%%%%%%%%%%%%%%%%%%%%%%%%%%%%%%%%%%%%%%%%%%%%%%%%%%%%%%%%%%%%%%%%
\begin{frame}[fragile]
\frametitle{Diagonal Matrices}

Diagonal matrices have non-zero elements only on
the main diagonal.
\begin{equation*}
    C = \text{diag}\left( c_1, c_2, \ldots, c_n \right)
      = \begin{bmatrix}  c_1   &  0  & \cdots & 0 \\
                          0    & c_2 &        & 0 \\
                        \vdots &     & \ddots & \vdots \\
                          0    &  0  & \cdots & c_n
        \end{bmatrix}
\end{equation*}

The \textbf{diag} function is used to either create a
diagonal matrix from a vector, or and extract the diagonal entries of a
matrix.
\begin{lstlisting}[language=matlab]
    >> x = [1 -5  2  6];
    >> A = diag(x)
    A =
         1     0     0     0
         0    -5     0     0
         0     0     2     0
         0     0     0     6
\end{lstlisting}

% ----------------------
\end{frame}
%%%%%%%%%%%%%%%%%%%%%%%%%%%%%%%%%%%%%%%%%%%%%%%%%%%%%%%%%%%%%%%%%%%%%%%%
%%%%%%%%%%%%%%%%%%%%%%%%%%%%%%%%%%%%%%%%%%%%%%%%%%%%%%%%%%%%%%%%%%%%%%%%
\begin{frame}[fragile]
\frametitle{Diagonal Matrices}

The \textbf{diag} function can also be used to create a
matrix with elements only on a specified \emph{super}-diagonal
or \emph{sub}-diagonal.  Doing so requires using the
two-parameter form of \textbf{diag}:
\begin{lstlisting}[language=matlab]
>> diag([1 2 3],1)
ans =
     0     1     0     0
     0     0     2     0
     0     0     0     3
     0     0     0     0
>> diag([4 5 6],-1)
ans =
     0     0     0     0
     4     0     0     0
     0     5     0     0
     0     0     6     0

\end{lstlisting}


% ----------------------
\end{frame}
%%%%%%%%%%%%%%%%%%%%%%%%%%%%%%%%%%%%%%%%%%%%%%%%%%%%%%%%%%%%%%%%%%%%%%%%
%%%%%%%%%%%%%%%%%%%%%%%%%%%%%%%%%%%%%%%%%%%%%%%%%%%%%%%%%%%%%%%%%%%%%%%%
\begin{frame}
\frametitle{Identity Matrices}

An identity matrix is a square matrix with ones on the main diagonal.
\begin{equation*}
    I = \begin{bmatrix}1 & 0 & 0\\ 0 & 1 & 0 \\ 0 & 0 & 1 \end{bmatrix}
\end{equation*}

An identity matrix is special because
\begin{equation*}
    AI = A
    \ \ \ \ \ \text{and}\ \ \ \ \
    IA = A
\end{equation*}
for \emph{any} compatible matrix $A$.  This is like multiplying by
one in scalar arithmetic.


% ----------------------
\end{frame}
%%%%%%%%%%%%%%%%%%%%%%%%%%%%%%%%%%%%%%%%%%%%%%%%%%%%%%%%%%%%%%%%%%%%%%%%
%%%%%%%%%%%%%%%%%%%%%%%%%%%%%%%%%%%%%%%%%%%%%%%%%%%%%%%%%%%%%%%%%%%%%%%%
\begin{frame}[fragile]
\frametitle{Identity Matrices}

Identity matrices can be created with the built-in \textbf{eye} function.
\begin{lstlisting}[language=matlab]
>> I = eye(4)
I =
     1     0     0     0
     0     1     0     0
     0     0     1     0
     0     0     0     1
\end{lstlisting}

Sometimes $I_n$ is used to designate an identity matrix with $n$ rows
and $n$ columns.  For example,
\begin{equation*}
    I_4 = \begin{bmatrix}  1 & 0 & 0 & 0 \\
                           0 & 1 & 0 & 0 \\
                           0 & 0 & 1 & 0 \\
                           0 & 0 & 0 & 1 \end{bmatrix}
\end{equation*}

% ----------------------
\end{frame}
%%%%%%%%%%%%%%%%%%%%%%%%%%%%%%%%%%%%%%%%%%%%%%%%%%%%%%%%%%%%%%%%%%%%%%%%
%%%%%%%%%%%%%%%%%%%%%%%%%%%%%%%%%%%%%%%%%%%%%%%%%%%%%%%%%%%%%%%%%%%%%%%%
\begin{frame}[fragile]
\frametitle{Identity Matrices}

A non-square, \emph{identity-like} matrix can be created with the
two-parameter form of the \texttt{eye} function:
\begin{lstlisting}[language=matlab]
>> J = eye(3,5)
J =
     1     0     0     0     0
     0     1     0     0     0
     0     0     1     0     0

>> K = eye(4,2)
K =
     1     0
     0     1
     0     0
     0     0
\end{lstlisting}
\texttt{J} and \texttt{K} are \emph{not} identity matrices!


% ----------------------
\end{frame}
%%%%%%%%%%%%%%%%%%%%%%%%%%%%%%%%%%%%%%%%%%%%%%%%%%%%%%%%%%%%%%%%%%%%%%%%
%%%%%%%%%%%%%%%%%%%%%%%%%%%%%%%%%%%%%%%%%%%%%%%%%%%%%%%%%%%%%%%%%%%%%%%%
\begin{frame}
\frametitle{Functions to Create Special Matrices}

\begin{center}
    \renewcommand{\arraystretch}{1.3}
    \begin{tabular}{ll}
        Matrix      &  Matlab\ function  \\  \hline
        Diagonal    &  \texttt{diag}     \\
        Identity    & \texttt{eye}     \\
        Inverse     & \texttt{inv}
    \end{tabular}
\end{center}

% ----------------------
\end{frame}
%%%%%%%%%%%%%%%%%%%%%%%%%%%%%%%%%%%%%%%%%%%%%%%%%%%%%%%%%%%%%%%%%%%%%%%%
%%%%%%%%%%%%%%%%%%%%%%%%%%%%%%%%%%%%%%%%%%%%%%%%%%%%%%%%%%%%%%%%%%%%%%%%
\begin{frame}
\frametitle{Symmetric Matrices}

If $A=A^T$, then $A$ is called a \textit{symmetric} matrix.

\begin{equation*}
    \begin{bmatrix}
        \phantom{-}5 & -2           & -1 \\
        -2           & \phantom{-}6 & -1 \\
        -1           & -1           & \phantom{-}3
    \end{bmatrix}
\end{equation*}

\begin{block}{Note}
    $B = A^T A$ is symmetric for any (real) matrix $A$.
\end{block}

% ----------------------
\end{frame}
%%%%%%%%%%%%%%%%%%%%%%%%%%%%%%%%%%%%%%%%%%%%%%%%%%%%%%%%%%%%%%%%%%%%%%%%
%%%%%%%%%%%%%%%%%%%%%%%%%%%%%%%%%%%%%%%%%%%%%%%%%%%%%%%%%%%%%%%%%%%%%%%%
\begin{frame}
\frametitle{Tridiagonal Matrices}

\begin{equation*}
\left[\begin{array}{@{}rrrr@{}}
2 & -1 & 0 & 0\\
-1 & 2 & -1 & 0\\
0 & -1 & 2 & -1\\
0 & 0 & -1 & 2
\end{array}\right].
\end{equation*}

The diagonal elements need not be equal.  The general
form of a tridiagonal matrix is
\begin{equation*}
    A = \begin{bmatrix}
    a_1 & b_1 & & & & &\\
    c_2 & a_2 & b_2& & & &\\
    & c_3 & a_3 & b_3& & &\\
    & & \ddots & \ddots & \ddots& &\\
    \\
    & & & & c_{n-1}& a_{n-1}& b_{n-1}\\
    & & & & & c_n & a_n
    \end{bmatrix}
\end{equation*}




% ----------------------
\end{frame}
%%%%%%%%%%%%%%%%%%%%%%%%%%%%%%%%%%%%%%%%%%%%%%%%%%%%%%%%%%%%%%%%%%%%%%%%
\end{document}
