% Copyright Luke Olson 2009--2014
% This work is licensed under the Creative Commons
% Attribution-NonCommercial-NoDerivatives 4.0 International License. To view a
% copy of this license, visit http://creativecommons.org/licenses/by-nc-nd/4.0/.
%
\documentclass[10pt]{beamer}
%\documentclass[handout,10pt]{beamer}
%
\mode<presentation>
{
  \usetheme[secheader]{Boadilla}
  \usefonttheme[onlymath]{serif}
  \setbeamercovered{invisible}
  \usecolortheme{luke}
  %\setbeamercovered{transparent}
  %
}
\mode<handout>
{
  \usetheme[secheader]{Boadilla}
  \usefonttheme[onlymath]{serif}
  \setbeamercovered{invisible}
  \usecolortheme{luke2}
  %\setbeamercovered{transparent}
}
\usepackage{pgf,pgfarrows,pgfnodes,pgfautomata,pgfheaps,pgfshade}
\usepackage{pifont}
\usepackage{pxfonts}
\usepackage{eulervm}
\usepackage{listings}
%\usepackage{pgfpages}
%\pgfpagesuselayout{2 on 1}[letterpaper]
%
%
%%%%%%%%%%%%%%%%%%%%%%%%%%%%%%%%%%%%%%%%%%%%%%%%%%%%%%%%%%%%%%%%%%%%%%%%


%
%
%
\newcommand{\vb}{{\bf{b}}}
\newcommand{\ve}{{\bf{e}}}
\newcommand{\vg}{{\bf{g}}}
\newcommand{\vp}{{\bf{p}}}
\newcommand{\vr}{{\bf{r}}}
\newcommand{\vu}{{\bf{u}}}
\newcommand{\vx}{{\bf{x}}}
\newcommand{\vz}{{\bf{z}}}
\newcommand{\vA}{{\bf{A}}}
\newcommand{\vU}{{\bf{U}}}
\newcommand{\mO}{{\mathcal{O}}}
\newcommand{\mF}{{\mathcal{F}}}
\definecolor{mygray}{rgb}{0.95,0.95,0.95}
\lstset{
        language=matlab,
        numbers=left, numberstyle=\tiny, stepnumber=1, numbersep=5pt,
        basicstyle=\color{black}\ttfamily\small,
        commentstyle=\color{green}\ttfamily,
        keywordstyle=\color{blue}\ttfamily,
        stringstyle=\color{red}\ttfamily,
        showstringspaces=false,
        backgroundcolor=\color{mygray},
        breaklines,
}
\newcommand{\norm}[1]{{\ensuremath{{\|#1\|}}}}
\newcommand{\matdim}[2]{\ensuremath{#1\times#2}}
\newcommand{\rank}[1]{\ensuremath{\mathrm{rank}(#1)}}
\newcommand{\epsm}{\ensuremath{\varepsilon_m}}
\newcommand{\cmd}[1]{{\normalfont\ttfamily\bfseries#1}}

\author{L. Olson}
\institute[UIUC]
{Department of Computer Science\\
University of Illinois at Urbana-Champaign\\
\vspace{0.5cm}
}
%%%%%%%%%%%%%%%%%%%%%%%%%%%%%%%%%%%%%%%%%%%%%%%%%%%%%%%%%%%%%%%%%%%%%%%%
\pgfdeclareimage[height=0.5cm]{university-logo}{./figs/uiuclogo}
\logo{\pgfuseimage{university-logo}}
%%%%%%%%%%%%%%%%%%%%%%%%%%%%%%%%%%%%%%%%%%%%%%%%%%%%%%%%%%%%%%%%%%%%%%%%
\title[CS 357]{Lecture 6}
\subtitle{Gaussian Elimination}
\date{September 10, 2009}

\begin{document}
% -------------------------------------------------
\begin{frame}
  \titlepage
\end{frame}
% -------------------------------------------------

%%%%%%%%%%%%%%%%%%%%%%%%%%%%%%%%%%%%%%%%%%%%%%%%%%%%%%%%%%%%%%%%%%%%%%%%
\begin{frame}
\frametitle{Gaussian Elimination}

\begin{itemize}
\item   Solving Triangular Systems
\item   Gaussian Elimination Without Pivoting
	\begin{itemize}
        \item   Hand Calculations
		\item   Cartoon Version
		\item   Algorithm
	\end{itemize}
\item Elementary Elimination Matrices And LU Factorization
%\item   Gaussian Elimination with Pivoting
%	\begin{itemize}
%		\item   Row or Column Interchanges, or Both
%		\item   Implementation
%	\end{itemize}
%\item   Solving Systems with the Backslash Operator
\end{itemize}

% ----------------------
\end{frame}
%%%%%%%%%%%%%%%%%%%%%%%%%%%%%%%%%%%%%%%%%%%%%%%%%%%%%%%%%%%%%%%%%%%%%%%%
\begin{frame}
\frametitle{Gaussian Elimination}
Gaussian elimination is a mostly general method for solving square systems.\\
\vspace{0.5cm}
We will work with systems in their matrix form, such as
\begin{align*}
      x_1 + 3x_2 + 5x_3 &= 4\\
     9x_1 + 7x_2 + 8x_3 &= 6\\
     3x_1 + 2x_2 + 7x_3 &= 1,\\
\end{align*}
in its equivalent matrix form,
\begin{equation*}
    \begin{bmatrix} 1 & 3 & 5 \\
                    9 & 7 & 8 \\
                    3 & 2 & 7 \\
    \end{bmatrix}
    \begin{bmatrix} x_1 \\
                    x_2 \\
                    x_3 \\
    \end{bmatrix} =
    \begin{bmatrix} 4 \\
                    6 \\
                    1 \\
    \end{bmatrix}.
\end{equation*}
\end{frame}
%%%%%%%%%%%%%%%%%%%%%%%%%%%%%%%%%%%%%%%%%%%%%%%%%%%%%%%%%%%%%%%%%%%%%%%%
\begin{frame}
\frametitle{Triangular Systems}

The generic lower and upper triangular matrices are
\begin{equation*}
    L = \begin{bmatrix} l_{11} &   0    & \cdots &   0    \\
                        l_{21} & l_{22} &        &   0    \\
                        \vdots &        & \ddots & \vdots \\
                        l_{n1} &        & \cdots & l_{nn} \\
        \end{bmatrix}
\end{equation*}
and
\begin{equation*}
    U = \begin{bmatrix} u_{11} & u_{12} & \cdots & u_{1n} \\
                           0   & u_{22} &        & u_{2n}  \\
                        \vdots &        & \ddots & \vdots \\
                           0   &        & \cdots & u_{nn} \\
        \end{bmatrix}
\end{equation*}

The triangular systems
\begin{equation*}
    Ly = b \ \ \ \ \ \ \ \ \ Ux = c
\end{equation*}
are easily solved by \textbf{forward substitution} and
\textbf{backward substitution}, respectively

% ----------------------
\end{frame}
%%%%%%%%%%%%%%%%%%%%%%%%%%%%%%%%%%%%%%%%%%%%%%%%%%%%%%%%%%%%%%%%%%%%%%%%
\begin{frame}[fragile]
\frametitle{Solving Triangular Systems}

Solving for $x_1, x_2, \ldots, x_n$
for an upper triangular system is called \textbf{backward substitution}.
\begin{lstlisting}[mathescape,caption=backward substitution (page 270),label=algo:backsub]
  given $A$ (upper $\bigtriangleup$), $b$               
  $x_n = b_n/a_{nn}$           
  for $i=n-1\ldots 1$          
    $s = b_i$                  
    for $j = i+1\ldots n$    
      $s = s - a_{i,j}x_j$   
    end                      
    $x_i = s/a_{i,i}$          
  end                          
\end{lstlisting}

\onslide<2->{
Using forward or backward substitution is sometimes referred
to as performing a \textbf{triangular solve}.
}

% ----------------------
\end{frame}
%%%%%%%%%%%%%%%%%%%%%%%%%%%%%%%%%%%%%%%%%%%%%%%%%%%%%%%%%%%%%%%%%%%%%%%%
%%%%%%%%%%%%%%%%%%%%%%%%%%%%%%%%%%%%%%%%%%%%%%%%%%%%%%%%%%%%%%%%%%%%%%%%
\begin{frame}
\frametitle{Operations?}
\begin{block}{cheap!}
  \begin{itemize}
    \item begin in the bottom corner: 1 div
    \item row -2: 1 mult, 1 add, 1 div, or 3 FLOPS
    \item row -3: 2 mult, 2 add, 1 div, or 5 FLOPS
    \item row -4: 3 mult, 3 add, 1 div, or 7 FLOPS
    \item $\vdots$
    \item row -$j$: about $2j-1$ FLOPS
  \end{itemize}
  Total FLOPS? $\sum_{j=1}^{n} 2j-1 = 2 \frac{ n(n+1)}{2}-n$ or $\mO(n^2)$ FLOPS
\end{block}
\end{frame}
%%%%%%%%%%%%%%%%%%%%%%%%%%%%%%%%%%%%%%%%%%%%%%%%%%%%%%%%%%%%%%%%%%%%%%%%
%%%%%%%%%%%%%%%%%%%%%%%%%%%%%%%%%%%%%%%%%%%%%%%%%%%%%%%%%%%%%%%%%%%%%%%%
\begin{frame}[fragile]
\frametitle{Gaussian Elimination}
\begin{itemize}
    \item Triangular systems are easy to solve in $\mO(n^2)$ FLOPS
    \item Goal is to transform an arbitrary, square system into an
          equivalent upper triangular system
    \item Then easily solve with backward substitution
\end{itemize}
\vspace{1.0cm}
This process is equivalent to the \emph{formal solution} of 
$Ax=b$, where $A$ is an \matdim{n}{n} matrix.
 \begin{equation*}
         x = A^{-1}b
 \end{equation*}
 \textbf{In MATLAB:}
 \begin{lstlisting}[language=matlab]
 >> A = ...
 >> b = ...
 >> x = A\b
 \end{lstlisting}


% ----------------------
\end{frame}
%%%%%%%%%%%%%%%%%%%%%%%%%%%%%%%%%%%%%%%%%%%%%%%%%%%%%%%%%%%%%%%%%%%%%%%%
%%%%%%%%%%%%%%%%%%%%%%%%%%%%%%%%%%%%%%%%%%%%%%%%%%%%%%%%%%%%%%%%%%%%%%%%
\begin{frame}
\frametitle{Gaussian Elimination --- Hand Calculations}

Solve
\begin{align*}
      x_1 + 3x_2 &= 5\\
    2 x_1 + 4x_2 &= 6
\end{align*}

Subtract 2 times the first equation from the second equation
\begin{align*}
      x_1 + 3x_2 &= 5\\
          - 2x_2 &= -4
\end{align*}
This equation is now in triangular form, and can be solved by
backward substitution.

% ----------------------
\end{frame}
%%%%%%%%%%%%%%%%%%%%%%%%%%%%%%%%%%%%%%%%%%%%%%%%%%%%%%%%%%%%%%%%%%%%%%%%
%%%%%%%%%%%%%%%%%%%%%%%%%%%%%%%%%%%%%%%%%%%%%%%%%%%%%%%%%%%%%%%%%%%%%%%%
\begin{frame}
\frametitle{Gaussian Elimination --- Hand Calculations}

The elimination phase transforms the matrix and right
hand side to an equivalent system
\begin{center}
\begin{minipage}[b]{0.4\textwidth}
\begin{align*}
      x_1 + 3x_2 &= 5\\
    2 x_1 + 4x_2 &= 6
\end{align*}
\end{minipage}
\begin{minipage}[c]{0.1\textwidth}
\raisebox{5ex}{$\longrightarrow$}
\end{minipage}
\begin{minipage}[b]{0.4\textwidth}
\begin{align*}
      x_1 + 3x_2 &= 5\\
          - 2x_2 &= -4
\end{align*}
\end{minipage}
\end{center}
The two systems have the same solution.  The right hand
system is upper triangular.

Solve the second equation for $x_2$
\begin{equation*}
    x_2 = \frac{-4}{-2} = 2
\end{equation*}
Substitute the newly found value of $x_2$ into the
first equation and solve for $x_1$.
\begin{equation*}
    x_1 = 5 - (3)(2) = -1
\end{equation*}

% ----------------------
\end{frame}
%%%%%%%%%%%%%%%%%%%%%%%%%%%%%%%%%%%%%%%%%%%%%%%%%%%%%%%%%%%%%%%%%%%%%%%%
%%%%%%%%%%%%%%%%%%%%%%%%%%%%%%%%%%%%%%%%%%%%%%%%%%%%%%%%%%%%%%%%%%%%%%%%
\begin{frame}
\frametitle{Gaussian Elimination --- Hand Calculations}

When performing Gaussian Elimination by hand, we can avoid
copying the $x_i$ by using a shorthand notation.

For example, to solve:
\begin{equation*}
    A = \left[\negthickspace\begin{array}{rrr} -3 &  2 & -1  \\
                                                6 & -6 &  7 \\
                                                3 & -4 &  4 \end{array}\negthickspace\right]
    \ \ \ \ \ \ \ \ \
    b = \left[\negthickspace\begin{array}{r} -1 \\ -7 \\ -6 \end{array}\negthickspace\right]
\end{equation*}
Form the \emph{augmented} system
\begin{equation*}
    \tilde{A} = \left[ A\ \ b \right] =
      \left[
           \left.\negthickspace\begin{array}{rrr}  -3 &  2 & -1  \\
                                                    6 & -6 &  7 \\
                                                    3 & -4 &  4 \end{array}
           \right|\begin{array}{r} -1 \\ -7 \\ -6 \end{array}\negthickspace\right]
\end{equation*}
The vertical bar inside the augmented matrix is just
a reminder that the last column is the $b$ vector.


% ----------------------
\end{frame}
%%%%%%%%%%%%%%%%%%%%%%%%%%%%%%%%%%%%%%%%%%%%%%%%%%%%%%%%%%%%%%%%%%%%%%%%
%%%%%%%%%%%%%%%%%%%%%%%%%%%%%%%%%%%%%%%%%%%%%%%%%%%%%%%%%%%%%%%%%%%%%%%%
\begin{frame}
\frametitle{Gaussian Elimination --- Hand Calculations}

Add 2 times row 1 to row 2, and add (1 times) row 1 to row 3
\begin{equation*}
    \tilde{A}_{(1)} = \left[
           \left.\negthickspace\begin{array}{rrr}  -3 &  2 & -1 \\
                                                    0 & -2 &  5 \\
                                                    0 & -2 &  3 \end{array}
           \right|\begin{array}{r} -1 \\ -9 \\ -7 \end{array}\negthickspace\right]
\end{equation*}
Subtract (1 times) row 2 from row 3
\begin{equation*}
    \tilde{A}_{(2)} = \left[
           \left.\negthickspace\begin{array}{rrr}  -3 &  2 & -1 \\
                                                    0 & -2 &  5 \\
                                                    0 &  0 & -2 \end{array}
           \right|\begin{array}{r} -1 \\ -9 \\ 2 \end{array}\negthickspace\right]
\end{equation*}

% ----------------------
\end{frame}
%%%%%%%%%%%%%%%%%%%%%%%%%%%%%%%%%%%%%%%%%%%%%%%%%%%%%%%%%%%%%%%%%%%%%%%%
%%%%%%%%%%%%%%%%%%%%%%%%%%%%%%%%%%%%%%%%%%%%%%%%%%%%%%%%%%%%%%%%%%%%%%%%
\begin{frame}
\frametitle{Gaussian Elimination --- Hand Calculations}

The transformed system is now in upper triangular form
\begin{equation*}
    \tilde{A}_{(2)} = \left[
           \left.\negthickspace\begin{array}{rrr}  -3 &  2 & -1 \\
                                                    0 & -2 &  5 \\
                                                    0 &  0 & -2 \end{array}
           \right|\begin{array}{r} -1 \\ -9 \\ 2 \end{array}\negthickspace\right]
\end{equation*}
Solve by back substitution to get
\begin{align*}
    x_3 &= \frac{\ \,2}{-2} = -1                             \\[10pt]
    x_2 &= \frac{\ \,1}{-2}\left(-9 -5x_3 \right) = 2        \\[10pt]
    x_1 &= \frac{\ \,1}{-3}\left(-1 -2x_2 + x_3 \right) = 2  \\[10pt]
\end{align*}


% ----------------------
\end{frame}
%%%%%%%%%%%%%%%%%%%%%%%%%%%%%%%%%%%%%%%%%%%%%%%%%%%%%%%%%%%%%%%%%%%%%%%%
%%%%%%%%%%%%%%%%%%%%%%%%%%%%%%%%%%%%%%%%%%%%%%%%%%%%%%%%%%%%%%%%%%%%%%%%
\begin{frame}
\frametitle{Gaussian Elimination --- Cartoon Version}

Start with the augmented system
\begin{center}
    \begin{tabular}{ccccc}
        $\begin{bmatrix}   x  & x & x & x    & x \\
                           x  & x  & x  & x  & x  \\
                           x  & x  & x  & x  & x \\
                           x  & x  & x  & x  & x \end{bmatrix}$
     \end{tabular}
\end{center}
The $x$'s represent numbers, they are generally \emph{not} the same
values.\\
\vspace{0.5cm}
Begin elimination using the first row as the \emph{pivot row}
and the first element of the first row as the pivot
element
\begin{center}
    \begin{tabular}{ccccc}
        $\begin{bmatrix}\boxed{x} & x & x & x & x \\
                           x  & x  & x  & x  & x  \\
                           x  & x  & x  & x  & x \\
                           x  & x  & x  & x  & x \end{bmatrix}$
     \end{tabular}
\end{center}

% ----------------------
\end{frame}
%%%%%%%%%%%%%%%%%%%%%%%%%%%%%%%%%%%%%%%%%%%%%%%%%%%%%%%%%%%%%%%%%%%%%%%%
%%%%%%%%%%%%%%%%%%%%%%%%%%%%%%%%%%%%%%%%%%%%%%%%%%%%%%%%%%%%%%%%%%%%%%%%
\begin{frame}[shrink]
\frametitle{Gaussian Elimination --- Cartoon Version}
\begin{itemize}
    \item Eliminate elements under the pivot element in the first column.
    \item $x'$ indicates a value that has been changed once.
\end{itemize}
\begin{center}
	\renewcommand{\arraystretch}{1.3}
	\footnotesize
	\begin{tabular}{ccccc}
		$\begin{bmatrix}\boxed{x} & x & x & x & x \\
						   x  & x  & x  & x  & x  \\
						   x  & x  & x  & x  & x \\
						   x  & x  & x  & x  & x \end{bmatrix}$
		& $\longrightarrow$ &
		$\begin{bmatrix}\boxed{x} & x & x & x & x \\
						   0  & x' & x' & x' & x' \\
						   x  & x  & x  & x  & x \\
						   x  & x  & x  & x  & x \end{bmatrix}$ \\[55pt]
		& $\longrightarrow$ &
		$\begin{bmatrix}\boxed{x} & x & x & x & x \\
						   0  & x' & x' & x' & x' \\
						   0  & x' & x' & x' & x' \\
						   x  & x  & x  & x  & x \end{bmatrix}$ \\[55pt]
		& $\longrightarrow$ &
		$\begin{bmatrix}\boxed{x} & x & x & x & x \\
						   0  & x' & x' & x' & x' \\
						   0  & x' & x' & x' & x' \\
						   0  & x' & x' & x' & x'\end{bmatrix}$ &
	\end{tabular}
\end{center}

% ----------------------
\end{frame}
%%%%%%%%%%%%%%%%%%%%%%%%%%%%%%%%%%%%%%%%%%%%%%%%%%%%%%%%%%%%%%%%%%%%%%%%
%%%%%%%%%%%%%%%%%%%%%%%%%%%%%%%%%%%%%%%%%%%%%%%%%%%%%%%%%%%%%%%%%%%%%%%%
\begin{frame}
\frametitle{Gaussian Elimination --- Cartoon Version}
\begin{itemize}
    \item The pivot element is now the diagonal element in the second row.
    \item Eliminate elements under the pivot element in the second column.
    \item $x''$ indicates a value that has been changed twice.
\end{itemize}
\begin{center}
	\renewcommand{\arraystretch}{1.3}
	\footnotesize
	\begin{tabular}{ccccc}
		$\begin{bmatrix} x &   x        & x & x & x    \\
						 0 & \boxed{x'} & x' & x' & x' \\
						 0 &   x'       & x' & x' & x' \\
						 0 &   x'       & x' & x' & x' \end{bmatrix}$
		& $\longrightarrow$ &
		$\begin{bmatrix} x &   x        & x   & x   & x   \\
						 0 & \boxed{x'} & x'  & x'  & x'  \\
						 0 &   0        & x'' & x'' & x'' \\
						 0 &   x'       & x'  & x'  & x'\end{bmatrix}$ \\[55pt]
		& $\longrightarrow$ &
		$\begin{bmatrix} x &   x        & x   & x   & x   \\
						 0 & \boxed{x'} & x'  & x'  & x'  \\
						 0 &   0        & x'' & x'' & x'' \\
						 0 &   0        & x'' & x'' & x'' \end{bmatrix}$
	\end{tabular}
\end{center}

% ----------------------
\end{frame}
%%%%%%%%%%%%%%%%%%%%%%%%%%%%%%%%%%%%%%%%%%%%%%%%%%%%%%%%%%%%%%%%%%%%%%%%
%%%%%%%%%%%%%%%%%%%%%%%%%%%%%%%%%%%%%%%%%%%%%%%%%%%%%%%%%%%%%%%%%%%%%%%%
\begin{frame}
\frametitle{Gaussian Elimination --- Cartoon Version}
\begin{itemize}
    \item The pivot element is now the diagonal element in the third row.
    \item Eliminate elements under the pivot element in the third column.
    \item $x'''$ indicates a value that has been changed three times.
\end{itemize}
\begin{center}
	\renewcommand{\arraystretch}{1.3}
	\footnotesize
	\begin{tabular}{ccccc}
		$\begin{bmatrix} x & x  &     x       & x   & x   \\
						 0 & x' &     x'      & x'  & x'  \\
						 0 & 0  & \boxed{x''} & x'' & x'' \\
						 0 & 0  &     x''     & x'' & x'' \end{bmatrix}$ 
		& $\longrightarrow$ &
		$\begin{bmatrix} x & x  &     x       & x    & x   \\
						 0 & x' &     x'      & x'   & x'  \\
						 0 & 0  & \boxed{x''} & x''  & x'' \\
						 0 & 0  &     0       & x''' & x''' \end{bmatrix}$ 
	\end{tabular}
\end{center}

% ----------------------
\end{frame}
%%%%%%%%%%%%%%%%%%%%%%%%%%%%%%%%%%%%%%%%%%%%%%%%%%%%%%%%%%%%%%%%%%%%%%%%
%%%%%%%%%%%%%%%%%%%%%%%%%%%%%%%%%%%%%%%%%%%%%%%%%%%%%%%%%%%%%%%%%%%%%%%%
\begin{frame}
\frametitle{Gaussian Elimination --- Cartoon Version}

\textbf{Summary}
\vspace{0.0cm}
\begin{itemize}
\item	Gaussian Elimination is an orderly process for transforming
		an augmented matrix into an equivalent upper triangular form.
\item	The elimination operation at the $k^{th}$ step is
\begin{equation*}
			\tilde{a}_{ij} = \tilde{a}_{ij} - (\tilde{a}_{ik}/\tilde{a}_{kk})\tilde{a}_{kj},\quad i>k, \quad j\ge k
\end{equation*}
%% WDG - I changed this to match the form on slide 18
\item	Elimination requires three nested loops.
\item	The result of the elimination phase is represented by the image below.
\end{itemize}
\begin{center}
	\renewcommand{\arraystretch}{1.3}
	\small
	\begin{tabular}{ccccc}
        $\begin{bmatrix}   x  & x & x & x & x \\
                           x  & x  & x  & x  & x  \\
                           x  & x  & x  & x  & x \\
                           x  & x  & x  & x  & x \end{bmatrix}$
		& $\longrightarrow$ &
		$\begin{bmatrix} x & x  &  x   & x    & x   \\
						 0 & x' &  x'  & x'   & x'  \\
						 0 & 0  &  x'' & x''  & x'' \\
						 0 & 0  &  0   & x''' & x''' \end{bmatrix}$ 
	\end{tabular}
\end{center}



% ----------------------
\end{frame}
%%%%%%%%%%%%%%%%%%%%%%%%%%%%%%%%%%%%%%%%%%%%%%%%%%%%%%%%%%%%%%%%%%%%%%%%
\begin{frame}
  \frametitle{Gaussian Elimination}
\textbf{Summary}
\begin{itemize}
  \item Transform a linear system into (upper) triangular form.  i.e.
transform lower triangular part to zero
\item Transformation is done by taking linear combinations of rows
\item Example: $a=\begin{bmatrix}a_1\\ a_2\end{bmatrix}$
\item If $a_1 \ne 0$, then
\[
\begin{bmatrix}
1 & 0 \\
-a_2 / a_1 & 1\\
\end{bmatrix}
\begin{bmatrix}
a_1 \\
a_2\\
\end{bmatrix}
=
\begin{bmatrix}
a_1 \\
0\\
\end{bmatrix}
\]
\end{itemize}
\end{frame}
%%%%%%%%%%%%%%%%%%%%%%%%%%%%%%%%%%%%%%%%%%%%%%%%%%%%%%%%%%%%%%%%%%%%%%%%

\begin{frame}[fragile]
\frametitle{Gaussian Elimination Algorithm}

\begin{lstlisting}[mathescape,caption=Forward Elimination
beta,label=algo:fwdelimb]
  given $A$, $b$
                               
  for $k=1\dots n-1$          
    for $i = k+1\ldots n$      
      for $j = k\ldots n$      
        $a_{ij} = a_{ij} - (a_{ik}/a_{kk}) a_{kj}$
      end                      
      $b_{i} = b_{i} - (a_{ik}/a_{kk}) b_{k}$
    end                        
  end                          
\end{lstlisting}
\vspace{0.3cm}
\begin{itemize}
  \item the multiplier can be moved outside the $j$-loop
  \item no reason to actually compute 0
\end{itemize}
\begin{alertblock}{}
Challenge: The loops over $i$ and $j$ may be exchanged---why would one be 
preferable?
\end{alertblock}
\vfill

% ----------------------
\end{frame}
%%%%%%%%%%%%%%%%%%%%%%%%%%%%%%%%%%%%%%%%%%%%%%%%%%%%%%%%%%%%%%%%%%%%%%%%
%%%%%%%%%%%%%%%%%%%%%%%%%%%%%%%%%%%%%%%%%%%%%%%%%%%%%%%%%%%%%%%%%%%%%%%%
\begin{frame}[fragile]
\frametitle{Gaussian Elimination Algorithm}

\begin{lstlisting}[mathescape,caption=Forward Elimination,label=algo:fwdelim]
  given $A$, $b$
                               
  for $k=1\dots n-1$          
    for $i = k+1\ldots n$      
      $xmult = a_{ik}/a_{kk}$
      $a_{ik} = xmult$
      for $j = k+1\ldots n$      
        $a_{ij} = a_{ij} - (xmult) a_{kj}$
      end                      
      $b_{i} = b_{i} - (xmult) b_{k}$
    end                        
  end                          
\end{lstlisting}
\vfill

% ----------------------
\end{frame}
%%%%%%%%%%%%%%%%%%%%%%%%%%%%%%%%%%%%%%%%%%%%%%%%%%%%%%%%%%%%%%%%%%%%%%%%
\begin{frame}[fragile]
\frametitle{Naive Gaussian Elimination Algorithm}

\begin{itemize}
  \item Forward Elimination
  \item + Backward substitution
  \item = Naive Gaussian Elimination
\end{itemize}
\vfill

\begin{example}
GE\_naive.m
GE\_naive\_test.m
\end{example}
% ----------------------
\end{frame}
%%%%%%%%%%%%%%%%%%%%%%%%%%%%%%%%%%%%%%%%%%%%%%%%%%%%%%%%%%%%%%%%%%%%%%%%
\begin{frame}
\frametitle{Forward Elimination Cost?}
What is the cost in converting from $A$ to $U$?
\begin{center}
\begin{tabular}{c c c c}\hline
Step & Add & Multiply & Divide\\\hline
1 & $(n-1)^2$ & $(n-1)^2$ & $n-1$\\
2 & $(n-2)^2$ & $(n-2)^2$ & $n-2$\\
\vdots & & & \\
n-1 & 1 & 1 & 1\\\hline
\end{tabular}
\end{center}
or
\begin{center}
\begin{tabular}{c c}\hline
add & $\sum_{j=1}^{n-1} j^2$ \\
multiply & $\sum_{j=1}^{n-1} j^2$ \\
divide & $\sum_{j=1}^{n-1} j$ \\\hline
\end{tabular}
\end{center}

\end{frame}
%%%%%%%%%%%%%%%%%%%%%%%%%%%%%%%%%%%%%%%%%%%%%%%%%%%%%%%%%%%%%%%%%%%%%%%%
\begin{frame}
\frametitle{Forward Elimination Cost?}
\begin{center}
\begin{tabular}{c c}\hline
add & $\sum_{j=1}^{n-1} j^2$ \\
multiply & $\sum_{j=1}^{n-1} j^2$ \\
divide & $\sum_{j=1}^{n-1} j$ \\\hline
\end{tabular}
\vspace{1cm}
\end{center}
We know $\sum_{j=1}^{p} j = \frac{p(p+1)}{2}$ and
$\sum_{j=1}^{p} j^2= \frac{p(p+1)(2p+1)}{6}$, so
\vspace{1cm}
\begin{center}
\begin{tabular}{c c}\hline
add-subtracts & $\frac{n(n-1)(2n-1)}{6}$\\
multiply-divides & $\frac{n(n-1)(2n-1)}{6} +
\frac{n(n-1)}{2}=\frac{n(n^2 - 1)}{3}$\\\hline
\end{tabular}
\end{center}

% ----------------------
\end{frame}
%%%%%%%%%%%%%%%%%%%%%%%%%%%%%%%%%%%%%%%%%%%%%%%%%%%%%%%%%%%%%%%%%%%%%%%%
\begin{frame}
\frametitle{Forward Elimination Cost?}
\begin{center}
\begin{tabular}{c c}\hline
add-subtracts & $\frac{n(n-1)(2n-1)}{6}$\\
multiply-divides & $\frac{n(n^2 - 1)}{3}$\\
add-subtract for $b$ & $\frac{n(n - 1)}{2}$\\
multipply-divides for $b$ & $\frac{n(n - 1)}{2}$\\\hline
\end{tabular}
\end{center}
% ----------------------
\end{frame}
%%%%%%%%%%%%%%%%%%%%%%%%%%%%%%%%%%%%%%%%%%%%%%%%%%%%%%%%%%%%%%%%%%%%%%%%
\begin{frame}
\frametitle{Back Substitution Cost}
As before
\begin{center}
\begin{tabular}{c c}\hline
add-subtract & $\frac{n(n - 1)}{2}$\\
multipply-divides & $\frac{n(n + 1)}{2}$\\\hline
\end{tabular}
\end{center}
% ----------------------
\end{frame}
%%%%%%%%%%%%%%%%%%%%%%%%%%%%%%%%%%%%%%%%%%%%%%%%%%%%%%%%%%%%%%%%%%%%%%%%
\begin{frame}
\frametitle{Naive Gaussian Elimination Cost}
Combining the cost of forward elimination and backward substitution
gives
\begin{center}
\begin{tabular}{c c}\hline
add-subtracts & $\frac{n(n-1)(2n-1)}{6} + \frac{n(n - 1)}{2} + \frac{n(n - 1)}{2}$\\
&  $=\frac{n(n-1)(2n+5)}{3}$\\
multiply-divides & $\frac{n(n^2 - 1)}{3}+\frac{n(n - 1)}{2}+ \frac{n(n + 1)}{2}$\\
& $=\frac{n(n^2 + 3n-1)}{3}$\\\hline
\end{tabular}
\end{center}
So the total cost of add-subtract-multiply-divide is about
\begin{equation*}
  \frac{2}{3} n^3
\end{equation*}
% ----------------------
$\Rightarrow$ double $n$ results in a cost increase of a factor of 8
\end{frame}
%%%%%%%%%%%%%%%%%%%%%%%%%%%%%%%%%%%%%%%%%%%%%%%%%%%%%%%%%%%%%%%%%%%%%%%%
\begin{frame}
\frametitle{Elimination Matrices}
\begin{itemize}
\item Another way to zero out entries in a column of $A$
\item Annihilate entries below $k^{th}$ element in $a$ with matrix, $M_k$:
\[
M_k a = 
\begin{bmatrix}
1 & \dots & 0 & 0 & \dots & 0\\
\vdots & \ddots & \vdots & \vdots & \ddots & \vdots \\
0 & \dots & 1 & 0 & \dots & 0\\
0 & \dots & -m_{k+1} & 1 & \dots & 0\\
\vdots & \ddots & \vdots & \vdots & \ddots & \vdots \\
0 & \dots & -m_{n} & 0 & \dots & 1\\
\end{bmatrix}
\begin{bmatrix}
a_1 \\ \vdots \\ a_k \\ a_{k+1}\\ \vdots \\ a_n
\end{bmatrix}
=
\begin{bmatrix}
a_1 \\ \vdots \\ a_k \\ 0\\ \vdots \\ 0
\end{bmatrix}
\]
where $m_i = a_i/a_k$, $i=k+1,\dots,n$.
\item The divisor $a_k$ is the ``pivot'' (and needs to be nonzero)
\end{itemize}
\end{frame}
%%%%%%%%%%%%%%%%%%%%%%%%%%%%%%%%%%%%%%%%%%%%%%%%%%%%%%%%%%%%%%%%%%%%%%%%
\begin{frame}\frametitle{Elimination Matrices}
\begin{itemize}
  \item Matrix $M_k$ is an ``elementary elimination matrix''
    \begin{itemize}
        \item Adds a multiple of row $k$ to each subsequent row, with ``multipliers'' $m_i$ 
        \item Result is zeros in the $k^{th}$ column for rows $i>k$.
    \end{itemize}
  \item $M_k$ is unit lower triangular and nonsingular
  \item $M_k = I - m_k e_k^T$ where $m_k = [0,\dots,0,m_{k+1},\dots,m_n]^T$ and
$e_k$ is the $k^{th}$ column of the identity matrix $I$.

  \item $M^{-1}_k = I + m_k e_k^T$, which means $M_k^{-1}$ is also lower triangular, and we will denote $M_k^{-1} = L_k$.
\end{itemize}
\begin{alertblock}{}
Can you prove $M^{-1}_k = I + m_k e_k^T$?
\end{alertblock}
\end{frame}
\begin{frame}\frametitle{Elimination Matrices}
\begin{itemize}
    \item Suppose $M_j$ and $M_k$ are elementary elimination matrices with $j>k$,
    then
    \begin{align*}
      M_k M_j & = I - m_k e_k^T - m_j e_j^T + m_k e_k^T m_j e_j^T\\
    & = I - m_k e_k^T - m_j e_j^T + m_k (e_k^T m_j) e_j^T\\
    & = I - m_k e_k^T - m_j e_j^T
    \end{align*}
    because the $k^{th}$ entry of vector $m_j$ is zero (since $j>k$)
    \item Thus $M_k M_j$ is essentially a union of their columns.
    \item Note this is also true for $M_k^{-1}M_j^{-1}$.
\end{itemize}
\end{frame}
\begin{frame}\frametitle{Example}
Let $a=\begin{bmatrix}2 \\ 4 \\ -2\end{bmatrix}$.
\[
M_1 a =
\begin{bmatrix}
  1 & 0 & 0\\
-2 & 1 & 0\\
1 & 0 & 1\\
\end{bmatrix}
\begin{bmatrix}
2 \\ 4 \\ -2
\end{bmatrix}
     =
\begin{bmatrix}
2 \\ 0 \\ 0
\end{bmatrix}
\]
and
\[
M_2 a =
\begin{bmatrix}
  1 & 0 & 0\\
0 & 1 & 0\\
0 & 1/2 & 1\\
\end{bmatrix}
\begin{bmatrix}
2 \\ 4 \\ -2
\end{bmatrix}
      =
\begin{bmatrix}
2 \\ 4 \\ 0
\end{bmatrix}
\]
\end{frame}
\begin{frame}\frametitle{Example}
  So 
\[
L_1 = M_1^{-1} = 
\begin{bmatrix}
  1 & 0 & 0\\
2 & 1 & 0\\
-1 & 0 & 1\\
\end{bmatrix},\,\,
L_2 = M_2^{-1} = 
\begin{bmatrix}
  1 & 0 & 0\\
0 & 1 & 0\\
0 & -1/2 & 1\\
\end{bmatrix}
\]
which means
\[
M_1 M_2 =
\begin{bmatrix}
  1 & 0 & 0\\
-2 & 1 & 0\\
1 & 1/2 & 1\\
\end{bmatrix},\,\,
L_1 L_2 =
\begin{bmatrix}
  1 & 0 & 0\\
2 & 1 & 0\\
-1 & -1/2 & 1\\
\end{bmatrix}
\]
\end{frame}
\begin{frame}
\frametitle{Gaussian Elimination}
\begin{itemize}
  \item To reduce $Ax=b$ to upper triangular form, first construct $M_1$ with
$a_{11}$ as the pivot (eliminating the first column of $A$ below the diagonal.)
  \item Then $M_1 A x = M_1 b$ still has the same solution.\
  \item Next construct $M_2$ with pivot $a_{22}$ to eliminate the second column
below the diagonal.
  \item Then $M_2 M_1 A x = M_2 M_1 b$ still has the same solution
  \item $M_{n-1} \dots M_1 A x = M_{n-1}\dots M_1 b$
  \item Let $M = M_{n}M_{n-1}\dots M_1$.  Then $M A x = M b$, with $MA$ upper triangular. 
  \item Do back substitution on $M A x = M b$.
\end{itemize}
\end{frame}

\begin{frame}
\frametitle{Another Way to Look at $A$}
We've mentioned $L$ and $U$ today.  Why?

Consider this
\begin{align*}
 A & = A\\
 A & = (M^{-1}M) A\\
 A & = (M_1^{-1}M_2^{-1}\dots M_n^{-1})(M_n M_{n-1} \dots M_1)A\\
 A & = (M_1^{-1}M_2^{-1}\dots M_n^{-1}) ((M_n M_{n-1} \dots M_1)A)\\
 A & = \phantom{(M_1^{-1}M_2^{-1}} L \phantom{M_n^{-1}) ((M_n M_{n-1}} U \phantom{M_1)A)}\\
\end{align*}
But $MA$ is upper triangular, and we've seen that $M_1^{-1}\dots M_n^{-1}$ is 
lower triangular.  Thus, we have an algorithm that factors $A$ into two 
matrices $L$ and $U$.
\end{frame}

\begin{frame}
\frametitle{Why is this ``naive''?}
\begin{example}
\begin{equation*}
A=
\begin{bmatrix}
  0 & 2 & 3\\
  4 & 5 & 6\\
  7 & 8 & 9\\
\end{bmatrix}
\end{equation*}
\end{example}
\begin{example}
\begin{equation*}
A=
\begin{bmatrix}
  1e-10 & 2 & 3\\
  4 & 5 & 6\\
  7 & 8 & 9\\
\end{bmatrix}
\end{equation*}
\end{example}

\end{frame}
\end{document}
